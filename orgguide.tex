% Created 2024-06-19 Wed 12:11
% Intended LaTeX compiler: pdflatex
\documentclass[11pt]{article}
\usepackage[utf8]{inputenc}
\usepackage[T1]{fontenc}
\usepackage{graphicx}
\usepackage{longtable}
\usepackage{wrapfig}
\usepackage{rotating}
\usepackage[normalem]{ulem}
\usepackage{amsmath}
\usepackage{amssymb}
\usepackage{capt-of}
\usepackage{hyperref}
\author{The Org Mode Developers Luca Canali}
\date{\today}
\title{Org Mode Compact Guide\\\medskip
\large Release 9.7}
\hypersetup{
 pdfauthor={The Org Mode Developers Luca Canali},
 pdftitle={Org Mode Compact Guide},
 pdfkeywords={},
 pdfsubject={},
 pdfcreator={Emacs 29.3 (Org mode 9.6.15)}, 
 pdflang={Italian}}
\begin{document}

\maketitle
\tableofcontents


\section{Copying}
\label{sec:orgd8c88d7}
Copyright \textcopyright{} 2004--2024  Free Software Foundation, Inc.

\begin{quote}
Permission is granted to copy, distribute and/or modify this document
under the terms of the GNU Free Documentation License, Version 1.3 or
any later version published by the Free Software Foundation; with no
Invariant Sections, with the Front-Cover Texts being ``A GNU Manual,''
and with the Back-Cover Texts as in (a) below.  A copy of the license
is included in the section entitled ``GNU Free Documentation License.''
in the full Org manual, which is distributed together with this
compact guide.

(a) The FSF's Back-Cover Text is: ``You have the freedom to copy and
modify this GNU manual.''
\end{quote}

\section{Introduzione}
\label{sec:orgf679320}
Org é una modalità per prendere note, mantenere liste TODO ed a pianificare
proggetti con un sistema di testo semplice veloce ed effettivo. È anche un
sistema di creazione e publicazione e supporta il lavoro con i blocchi di
codice per la programmazione letterale e la ricerca riproducibile.

Questo documento é un derivato molto compresso dal più completo \href{https://www.gnu.org/software/emacs/manual/html\_mono/org.html\#Top}{manuale Org
mode}. Contiene tutte le funzionalità ed i comandi di base, assieme a
importanti suggerimenti per la personalizzazione. È destinato ai
principianti che potrebbero tirarsi indietro, a causa delle dimensioni,
davanti ad un manuale di 200 pagine.

\subsection*{Installazione}
\label{sec:org61fede0}
\begin{quote}
Se usi una versione di Org che fa parte della distribuzione di Emacs,
perfavore salta questa sezione e vai direttamente a \hyperref[sec:org316fc0b]{Attivazione}.
\end{quote}

Se hai scaricato Org dal web, solitamente viene distribuito come file \texttt{.zip}
o \texttt{.tar} oppure come archivio Git, é meglio avviarlo direttamente dalla
directory di distribuzione. Hai bisogno di aggiungere la sottodirectory
\texttt{lips/} alla path di Emacs. Per fare questo aggiungi la seguente linea nel
tuo file di inizializzazione:

\begin{verbatim}
(add-to-list 'load-path "~/path/to/orgdir/lisp")
\end{verbatim}


Se hai usato git o tar ball per ottenere Org, hai bisogno di avviare il
seguente comando per generare le informazioni di autoload.

\begin{verbatim}
make autoload
\end{verbatim}

\subsection*{Attivazione}
\label{sec:org316fc0b}
Aggiiungi le seguenti linee al tuo file di init per definire delle chiavi
\emph{globali} per tre comandi che sono utili in ogni buffer Emacs, non solo nel
buffer Org. Perfavore scegli le chiavi più comode per te.

\begin{verbatim}
(global-set-key (kbd "C-c l") #'org-store-link)
(global-set-key (kbd "C-c a") #'org-agenda)
(global-set-key (kbd "C-c c") #'org-capture)
\end{verbatim}

I file con estensione \texttt{.org} sono posti in modo automatico nella modalità
Org.

\subsection*{Feedback}
\label{sec:orgdc26732}
Se riscontri dei problemi con Org o se hai delle domande, osservazioni o
idee a riguardo, perfavore invia una mail alla mailing list
\url{mailto:emacs-orgmode@gnu.org}.  Per informazioni su come inviare dei report
sui bug, guarda il \href{org}{manuale pricipale}.

\section{Struttura del documento}
\label{sec:orga976bef}
Org è uno schematizzatore. La schematizzazione permette al documento di
essere organizzato in strutture gerarchiche, che, almeno per me, sono la
migliore rappresentazione delle note e dei pensieri. Un'osservazione di
questa struttura si ottiene piegando, es, nascondendo larghe porzioni del
documento per mostrarne solo la struttura generale e la parte dove al
momento si sta. Org semplifica molto bene l'uso delle outline comprimendo
l'intera vista e nascondendo le funzionalita in un singolo comando
\texttt{org-cycle} che é legato al tasto TAB.

\subsection{Intestazioni}
\label{sec:orgf5b2a10}
Le intestazioni definiscono la struttura di uno schema ad albero. Le
intestazioni in Org iniziano dal margine sinistro\footnote{Guarda la variabile \texttt{org-special-ctrl-a/e} per configurare
speciali comportamenti di C-a e C-e nei titoli.} con uno o più
asterischi seguiti da uno spazio. Per esempio:

\begin{verbatim}
* Livello iniziale dei titoli
** Secondo livello
*** Terzo livello
Un po di testo
*** Terzo livello
Altro testo
* Un'altro livello iniziale
\end{verbatim}

Nota che un'intestazione nominata dopo \texttt{org-footnote-section}, che di default
é detto \texttt{Footnotes}, é considerato come speciale. Un sottoalbero con questa
intestazione sarà ignorata silenziosamente dalla funzione di esportazione.

Alcune persone trovano i vari asterischi troppo fastidiosi e preferiscono
uno schema con degli spazzi vuoti seguiti da un singolo asterisco. Guarda
nelle \ref{sec:org62c1089} per un setup per realizzarlo.

\subsection{Ciclo di visualizzazione}
\label{sec:org7e708e6}
Gli schemi rendono possibile nascondere parte del testo nel buffer. Org usa
giusto due comandi, legati aTAB e S-TAB (\texttt{org-cycle} e
\texttt{org-shifttab} ) per scambiare la visibilita nel buffer.

\begin{description}
\item[{TAB (\texttt{org-cycle})}] \emph{Ciclo del sottoalbero}: Ruota il sottoalbero corrente attraverso gli stai

\begin{verbatim}
,-> CHIUSO -> FIGLI -> SOTTOALBERO --.
'------------------------------------'
\end{verbatim}


Quando viene chiamato con l'argomento prefisso (C-u TAB) o con
il tasto Shitf, viene invocato il ciclo globale.

\item[{S-TAB (\texttt{org-global-cycle}), C-u TAB (\texttt{org-cycle})}] \emph{Ciclo globale}: Ruota l'intero buffer attraverso gli stati

\begin{verbatim}
,-> OVERVIEW -> CONTENUTO -> MOSTRA TUTTO --.
'-------------------------------------------'
\end{verbatim}

\item[{C-u C-u C-u TAB (\texttt{org-show-all})}] Mostra tutto incluso i drawer.

Quando Emacs visita per la prima volta un file Org, lo stato globle é
settato a OVERVIEW, es. solo i livelli più alti delle intestazione sono
visibili. Questo può essere configurato tramite la variabile
\texttt{org-startup-folder} o nell'intestazione del file aggiungendo una keyword
\texttt{STARTUP} con \texttt{overview}, \texttt{content}, \texttt{shoall}, \texttt{showeverything} o \texttt{show<n>levels} (n
= 2..5 ) tipo:

\begin{verbatim}
#+STARTUP: content
\end{verbatim}
\end{description}

\subsection{Movimeto}
\label{sec:org13b0ce1}
I comandi seguenti ti fanno saltare ad altri titoli nel buffer

\begin{description}
\item[{C-c C-n (\texttt{org-next-visible-heading})}] Titolo sucessivo
\item[{C-c C-p (\texttt{org-previous-visible-heading})}] Titolo precedente

\item[{C-c C-f (\texttt{org-backward-headling-same-level})}] Titolo sucessivo nello stesso livello

\item[{C-c C-b (\texttt{outline-backward-same-level})}] Titolo precedente nello stesso livello

\item[{C-c C-u (\texttt{outline-up-heading})}] Torna al titolo più altro
\end{description}

\subsection{Modificare la struttura}
\label{sec:org0b8d069}
\begin{description}
\item[{\textbf{M-RET} (\texttt{org-meta-return})}] Inserisce un nuovo titolo con lo stesso livello del corrente. Se il punto
é una lista di oggetti, viene creato un nuovo oggetto (vedere \ref{sec:orgd18e700}). Quando questo comando viene usato nel mezzo di una linea, la
linea viene divisa e il resto della linea diventa il nuovo titolo\footnote{Se non vuoi che la linea sia divisa, personalizza la variabile
\texttt{org-M-RET-may-split-line}}

\item[{M-S-RET (\texttt{org-inser-todo-heading})}] Inserisce un nuovo oggetto TODO con lo stesso livello del titolo corrente

\item[{TAB (\texttt{org-cycle}) in un oggetto vuoto}] In un nuovo oggetto che non ha ancora del testo, il TAB cicla tra i vari
livelli

\item[{M-SINISTRA (\texttt{org-metaleft}), M-DESTRA (\texttt{org-metaright})}] Promuove o retrocede il titolo corrente di un livello

\item[{M-SU (\texttt{org-move-subtree-up}), M-GIÙ (\texttt{org-move-subtree-down})}] Muove il sotttoalbero su o giù es. scambia con il precedente o il
successivo sottoalbero con lo stesso livello.

\item[{C-c C-w (\texttt{org-refile})}] Ricolloca un oggetto o una regione in una posizione differente. Vedere
\ref{sec:org800e315}.

\item[{C-x n s (\texttt{org-narrow-to-subtree}), C-x n w (\texttt{widen})}] Allarga il buffer al corrente sottoalbero e lo nasconde nuovamente
\end{description}

Quando c'é una regione attiva (Modalità Contrasegno di transito), promuove
e retrocede lavorando su tutti i titoli nella regione.

\subsection{Alberi sparsi}
\label{sec:org7974999}
Una funzionalità importante della modalità Org é l'abilità di costruire
\emph{alberi sparsi} per le informazioni selezionate in una struttura ad albero,
in modo che l'intero documento sia piegato il più possibile, ma le
informazioni selezionate siano comunque visibili assieme alla struttura
dell'intestazione sopra di esso\footnote{Vedi anche la variabile \texttt{org-show-context-detail} per decidere
quanto contesto mostrare attorno ad ogni selezione.}. Provalo subito e vedrai immediatamente
come lavora.

La modalita Org contiene diversi comandi che creano questi alberi, a tutti
questi comandi si può accedere attraverso uno dispacher.

\begin{description}
\item[{C-c / (\texttt{org-sparse-tree})}] Mostra un prompt in cui selezionare una chiave per creare un comando di
sparse-tree

\item[{C-c / r (\texttt{org-occur})}] Mostra un prompt per un'espressione regolare (regexp) e mostra un albero
sparso con tutte le occorrenze trovate. Ogni occorrenza viene
sottolineata; La sottolineatura svanisce se si preme C-c C-c.

L'altro comando di sparse tree seleziona le intestazioni basandosi sulle
chiavi dei TODO, tags o proprietà che verrà discusso più aventi in questo
documento.
\end{description}

\subsection{Liste semplici}
\label{sec:orgd18e700}
Assieme gli oggetti della struttura ad albero, gli elenchi formattati
manualmente possono fornire una struttura aggiuntiva. Inoltre
forniscono un modo per creare liste selezionabili o caselle di
controllo ( vedi [BROKEN LINK: *Checkboxes] ). Org supporta la modifica di queste
liste ed ogni esportatore ( Vedi \ref{sec:orgc8db748}) può analizzarlo e
formattarlo.

Org riconosce le liste ordinate, non ordinate e descritte.

\begin{itemize}
\item Gli elementi di una lista \emph{non ordinata} iniziano con un \texttt{-}, \texttt{+} o
un \texttt{*} come puntatore.
\item Gli elementi di una lista \emph{ordinata} iniziano con un \texttt{1.} o \texttt{1)}.
\item Le liste \emph{descritte} usano \texttt{::} per separare il \emph{termine} dalla
descrizione.
\end{itemize}

Gli elementi appartenenti allo stesso elenco devono avere la stessa
indentazione sulla prima riga. Un elemento termina prima della riga
successiva che é rientrato come il su elenco/numero o meno. Un elenco
termina quanto tutti gli elementi vengono chiusi o dopo due righe
vuote.

\begin{verbatim}
* Lord of the Rings
  My favorite scenes are (in this order)
  1. The attack of the Rohirrim
  2. Eowyn's fight with the witch king
     + this was already my favorite scene in the book
     + I really like Miranda Otto.
  Important actors in this film are:
  - Elijah Wood :: He plays Frodo
  - Sean Astin :: He plays Sam, Frodo's friend.
\end{verbatim}

I seguenti comandi agiscono sugli elementi quando il punto si trova
nella prima riga di un elemento (la riga con il punto elenco o il
numero).

\begin{description}
\item[{TAB (\texttt{org-cycle})}] Gli elementi possono essere ripiegati come le intestazioni.

\item[{M-RET (\texttt{org-insert-heading})}] Inserisce un nuovo elemento nel livello corrente. Con un argomento
prefisso, forza una nuova intestazione ( vedere [BROKEN LINK: Modificare le strutture] ).

\item[{M-S-RET (\texttt{org-insert-todo-heading})}] Inserisce un nuovo elemento con un casella di controllo ( vedere
[BROKEN LINK: *Checkboxes] ).

\item[{M-S-SU* (\texttt{org-move-item-up}), M-S-GIU (\texttt{org-move-item-down})}] Muove l'elemento, inclusi i sottoelementi, su/giù ( spostandoli con
il precedente/successivo elemento nella medesima indentazione ). Se
la lista é ordinata, la rinumera in automatico.

\item[{M-SINISTRA (\texttt{org-do-promote}), M-DESTRA (\texttt{org-do-subtree})}] Aumenta/diminuisce l'indentazione dell'elemento, fatta esclusione
per i figli.

\item[{M-S-SINISTRA (\texttt{org-promote-subtree}), M-S-DESTRA (\texttt{org-demote-subtree})}] Aumenta/diminuisce l'indentazione dell'elemento, esclusi i figli.

\item[{C-c C-c (\texttt{org-toggle-checkbox})}] Se c'é una casella di controllo ( vedere [BROKEN LINK: Checkboxes] ) nella linea
degli elementi, cambia lo stato della checkbox. Verificando anche i
puntatori e la consistenza del'indentazione in tutta la lista.

\item[{C-c - (\texttt{org-cycle-list-bullet})}] Scorre tra la lisata dei livelli degli elementi attraverso
differenti tipi di puntatori ( \texttt{-}, \texttt{+}, \texttt{*}, \texttt{1.}, \texttt{1)}).
\end{description}

\section{Tabelle}
\label{sec:org08f7d33}
Org ha un veloce ed intuitivo editor di tabelle. Come per i fogli di
calcolo sono supportati i calcoli, in connessione con il pacchetto
Emacs Calc (vedere \href{https://www.gnu.org/software/emacs/manual/html\_mono/calc.html\#Top}{GNU Emacs Calculator Manuale} ).

Org rende facile formattare le tabelle in semplice ACSII. Ogni linea
con \texttt{|} come primo carattere senza spazi viene considerato come parte
di una tabella. Il carattere \texttt{|} é anche un separatore di colonne. Una
tabella si presenta tipo questa:

\begin{verbatim}
| Name  | Phone | Age |
|-------+-------+-----|
| Peter |  1234 |  17 |
| Anna  | 43212 |  25 |
\end{verbatim}

Una tabella viene riallineata automaticamente ogni volta che viene
premuto TAB o RET oppure C-c C-c
quando ci si trova all'interno della tabella. Il TAB
inoltre muove il punto al campo successivo ( RET alla riga
successiva ) e crea nuove righe alla fine della tabella o prima delle
linee orrizontali. L'indentation di una tabella é settata nella prima
linea. Ogni linea che inizia con un \texttt{|-} é considerata un separatore
orrizzontale e viene espanso al successivo riallineamento per occupare
tutta la lareghezza della tabella. Quindi per creare la tabella
precedente devi solo digitare

\begin{verbatim}
|Name|Phone|Age|
|-
\end{verbatim}


e premere TAB per allineare la tabella ed iniziare a
riempire i campi. Per voler essere più veloce digita \texttt{|Name|Phone|Age}
seguito da C-c RET.

Quando digiti un testo in un campo, Org tratta DEL,
Backspace e tutti gli altri caratteri, in un modo speciale,
in modo che l'inserimento e la cancellazione non sposti gli altri
campi. Inoltre quando digiti \emph{subito dopo che il punto viene mosso in
un nuvo campo con TAB, S-TAB o RET},
il campo viene automaticamente creato bianco.

\subsection*{Creazione e conversione}
\label{sec:org30163b6}
\begin{description}
\item[{C-c | (\texttt{org-table-create-or-convert-from-regio})}] Converte la regione attiva in una tabella. Se ogni linea contiene
almeno un carattere TAB, la funzione assume che le cose
siano separato da tab. Se ogni linea é separata da virgola (CSV)
assume che sia un file separatoda virgole. Se no, le linee sono
divise in capi tramite gli spazzi bianchi.

Se non ci sono regioni attive, questo comando crea una tabella Org
vuota. Ma é più facile inserendo | N a m e | P h o n e | A g e RET | - TAB.
\end{description}

\subsection*{Ri-allineamento e spostamento dei campi}
\label{sec:orga223928}
\begin{description}
\item[{C-c C-c (\texttt{org-table-align})}] Ri-allinea la tabella senza spostare il punto.

\item[{TAB (\texttt{org-table-next-field})}] Ri-allinea la tabella, muovendosi nel campo successivo. Crea una
nuova riga se necessario.

\item[{S-TAB (\texttt{org-table-previous-field})}] Ri-allinea la tabella, mouvendo nel campo precedente.

\item[{RET (\texttt{org-table-next-row})}] Ri-allinea la tabella e muove in basso nella riga
successiva. Creando una nuova riga se necessario.

\item[{S-SU (\texttt{org-table-move-cell-up}), S-GIU (\texttt{org-table-move-cell-down}), S-SINISTRA (\texttt{org-table-move-cell-left}), S-DESTRA (\texttt{org-table-move-cell-right})}] Muove le celle su, giù, a sinistra e a destra spostando la cella adiacente.
\end{description}

\subsection*{Modifica colonne e righe}
\label{sec:org55611b8}
\begin{description}
\item[{M-SINISTRA (\texttt{org-table-move-column-left}), M-DESTRA (\texttt{org-table-move-column-right})}] Muove la colonna corrente a sinista/destra.

\item[{M-S-SINISTRA (\texttt{org-table-delete-column})}] Elimnina la colonna corrente.

\item[{M-S-DESTRA (\texttt{org-table-insert-column})}] Inserisce una nuova colonna a sinistra della posizione del punto.

\item[{M-SU (\texttt{org-tablet-move-row}), M-GIU (\texttt{org-tablet-move-row-down})}] Muove la riga corrente su/giù.

\item[{M-S-SU (\texttt{org-table-kill-row})}] Elimina la riga corrente o la linea orrizzontale.

\item[{M-S-GIU (\texttt{org-table-insert-row})}] Inserisce una nuova riga sopra la corrente. Con un argomento
prefisso la linea viene creata sotto la corrente.

\item[{C-c - (\texttt{org-table-insert-hline})}] Inserisce una linea sotto la corrente. Con un argomento prefisso la
linea viene creata sopra la corrente.

\item[{C-c RET (\texttt{org-table-hline-and-move})}] Inserise una line sotto la corrente e muove il punto nella riga
sotto questa linea.

\item[{C-c \^{} (\texttt{org-table-sort-lines})}] Ordina le linee della tabella in regioni. La posizione del punto
indica la colonna da usare per l'ordinamento ed il range delle linee
é il range delle linee tra le line orrizzontali più vicine o
l'intera tabella.
\end{description}

\section{Collegamenti ipertestuali}
\label{sec:org651b8f7}
Come nell'HTML, Org fornisce i link all'interno di un file, link
esterni ad altri file, articoli Usenet, email e moltro altro.

Org riconosce i semplici URIs, é possibile racchiuderli tra parentesi
quadrate ed attivarli come link cliccabili. Il formato generale dei
link si presenta simile a questo:

\begin{verbatim}
[[LINK][DESCRIZIONE]]
\end{verbatim}


od in alternativa

\begin{verbatim}
[[LINK]]
\end{verbatim}


Quando un link nel buffer é completo, con tutte le parentesi, Org
cambia la sua visualizzazione con \texttt{DESCRIZIONE} al posto di
\texttt{[[LINK]DESCRIZIONE]]} e in \texttt{LINK} al posto di \texttt{[[LINK]]}. Per
modificare la parte invisibile del LINK usa C-c C-l con il punto sul link.

\subsection*{Link interni}
\label{sec:org3026f41}
Se i link non sono rappresentati come URL, vengono considerati come
interni al file corrente. Il caso più importante é un link tipo
\texttt{[[\#my-custom-id]]} che viene collegata ad un elemento con la
proprietà \texttt{CUSTOM\_ID} settata a \texttt{my-custom-id}.

Link tipo \texttt{[[My Target]]} o \texttt{[[My Targhet][Find my target]]} va, nel
file corrente, alla ricerca di un testo che corrisponda al targhet
come \texttt{<{}<{}My Target>{}>{}}.

\subsection*{Link esterni}
\label{sec:org42b0785}
Org supporta i lin ai file, siti web, Usenet, messagi email, elementi
BBDB database ed sia le conversazioni che i logs IRC. I collegamenti
sono simili agli URL. Iniziano con una breve stringa di
indentificazione se quita dai duepunti \textbf{:}. Non ci sono spazzi dopo i
duepunti. Ecco alcuni esempi:

\begin{center}
\begin{tabular}{ll}
\texttt{http://www.astro.uva.nl/=dominikper} & Nel web\\[0pt]
\texttt{file:/home/dominik/images/jupiter.jpg} & file, path assoluta\\[0pt]
\texttt{/home/dominik/images/jupiter.jpg} & come sopra\\[0pt]
\texttt{file:papers/last.pdf} & file, path relativa\\[0pt]
\texttt{./papaers/last.pdf} & come sopra\\[0pt]
\texttt{file:project.org} & un'altro file org\\[0pt]
\texttt{docview:papers/last.pdf::NNN} & apre in modalità DocView alla pagina NNN\\[0pt]
\texttt{id:B7423F4D-2E8A-471B-8810-C40F074717E9} & link ad un titolo con ID\\[0pt]
\texttt{news:comp.emacs} & Usenet link\\[0pt]
\texttt{mailto:adent@galaxy.net} & link mail\\[0pt]
\texttt{mhe:folder\#id} & link messaggio MH-E\\[0pt]
\texttt{rmail:folder\#id} & link messaggio Rmail\\[0pt]
\texttt{gnus:gruop\#id} & link articolo Gnus\\[0pt]
\texttt{bbdb:R.*Stalman} & link BBDB (con regexp)\\[0pt]
\texttt{irc:/irc.com/\#emacs/bob} & link IRC\\[0pt]
\texttt{info:org\#Hyperlink} & link nodo Info\\[0pt]
\end{tabular}
\end{center}

I collegamento ai file possono contenere informazioni addizionali per
consetire ad Emacs di saltare ad una particolare posizione nel file
quando si segue un link. Questa può essere un numero di linea o un
opzione di ricerca dopo i due duepunti. Di seguito alcuni esempi,
assime alle spiegazioni:

\begin{center}
\begin{tabular}{ll}
\texttt{file:\textasciitilde{}/code/main.c::255} & Trova la linea 255\\[0pt]
\texttt{file:\textasciitilde{}/xx.org::My Target} & Trova \texttt{=<{}<{}My Target>{}>{}=}\\[0pt]
\texttt{[[file:\textasciitilde{}/xx.org::\#my-custom-id]]} & Trova l'elemento con l'ID personalizzato\\[0pt]
\end{tabular}
\end{center}

\subsection*{Manipolare i collegamenti}
\label{sec:org728b7ac}
Org fornisce un modo per creare link con la sintassi corretta, per
inserirla in un file Org e per seguire il link.

La funzione principale é \texttt{org-store-link} richiamabile con M-x org-store-link. Per la sua importanza, suggeriamo di abbinarlo ad
una combinazione più comoda ( vedere [BROKEN LINK: *Attivazioni] ). Salva un link
nella posizione attuale. Il link viene salvato per essere
successiavamente inserito in un buffer Org -- vedi sotto.

Da un buffer Org, i comandi seguenti creano, navigano o, più
generalmente, agiscono sui link.

\begin{description}
\item[{C-c C-l (\texttt{org-insert-link})}] Inserisce un link. Questo richiede un link che sarà inserito nel
buffer. Puoi inserire il link o usare la history SU e
GIU per accedere ai link salvati. Richiederà la parte di
descrizione per il link.

Quando chiamato con l'argomento perfisso C-u il
completamento dei nomi viene usato per l'inserimento di un nome di
file.

\item[{C-c C-l (Che punta ad un link esistente) (\texttt{org-insert-link})}] Quando il punto é un link esistente, C-c C-l permette la
modifica del link e della parte della descrizione.

\item[{C-c C-o (\texttt{open-link-at-point})}] Apre il link su cui si si trova.

\item[{C-c \& (\texttt{org-mark-ring-goto})}] Torna indietro a una posizione registrata. Una posizione é
registrata tramite il comando di link interno e tramite C-c \%. Usando questo comando più volte in diretta successione si
muove attraverso un anello di posizioni precedenti.
\end{description}

\section{Elementi TODO}
\label{sec:org309cf1d}
Org mode non necessita che le liste TODO risiedano in documenti
separati. Tantè che gli elementi TODO sono parte delle file di note,
perché gli elementi TODO vengono solitamente usati quando si prendono
note! Con la modalità Org, semplicemente si marca un oggeto in un
albero mettendo all'inizio un elemento TODO. In questo modo, le
informazioni non vengono duplicate e gli elementi TODO rimangono nel
contesto in modo che emergano.

La modalità Org ti fornisce dei metodi per una visione di insieme
delle cose che devi fare, raccogliendoli da diversi file.

\subsection[TODO Basics]{Funzionalità base dei TODO}
\label{sec:orgcd98aa1}
Un'intestazione é un elemento TODO quando inizia con la parola \texttt{TODO},
per esempio:

\begin{verbatim}
*** TODO Write letter to Sam Fortune
\end{verbatim}


I comandi più importanti quando si lavora con i TODO sono:

\begin{description}
\item[{C-c C-t (\texttt{org-todo})}] Cambia lo stato del TODO dell'elemento corrente

\begin{verbatim}
,-> (non segnato) -> TODO -> DONE --.
'-----------------------------------'
\end{verbatim}


La stessa rotazione può anche essere fatta ``da remoto'', dal buffer
dell'agenda, tramite la chiave t ( vedi \ref{sec:org15525c6} ).

\item[{S-RIGHT (\texttt{org-shiftright}), S-LEFT (\texttt{org-shiftleft})}] Seleziona il seguente/precedente stato TODO, similmente al cilco.

\item[{C-c / t (\texttt{org-show-todo-tree})}] Visualizza gli elementi TODO in un \emph{albero sparso} ( vedi \ref{sec:org7974999} ). Piega l'intero buffer ma mostra tutti gli elementi
TODO - con uno stato non-DONE - e la gerarchia delle intestazioni
sopra di loro.

\item[{M-x org-agenda t (\texttt{org-todo-list})}] Mostra la lista globale dei TODO. Colleziona tutti gli elementi TODO
( con stato non-DONE ) da tutti i file agenda ( vedi
\ref{sec:org3ad7d0f} ) in un unico buffer. Vedi \ref{sec:org0f74436} per maggiori informazioni.

\item[{S-M-RET (\texttt{org-insert-todo-heading})}] Inserisce un nuovo oggetto TODO sotto al corrente.
\end{description}

Cambiando lo stato di un TODO può anche cambiare i tag. Vedi la
docstring dell'opzinoe \texttt{org-todo-state-tags-triggers} per dettagli.

\subsection{Workflow multi-stato}
\label{sec:org10fc35f}
Puoi usare le parole chiave TODO per indicare un stato di lavoro
\emph{sequenziale}:

\begin{verbatim}
(setq org-todo-keywords
      '((sequence "TODO" "FEEDBACK" "VERIFY" "|" "DONE" "DELEGATED")))
\end{verbatim}

La barra verticale separa le keyword \texttt{TODO} ( stati che \emph{necessitano
azioni} ) da quelle \texttt{DONE} ( che \emph{non necessitano future azioni} ). Se
non fornisci una barra di separazione, l'ultimo stato viene usato come
stato \texttt{DONE}. Con questo setup, il comando C-c C-t cicla un
oggetto da \texttt{TODO} a \texttt{FEEDBACK} quindi \texttt{VERIFY} ed in fine \texttt{DONE} e
\texttt{DELEGATED}.

A volte vorresti usare, in parallelo, diversi insiemi di parole chiave
TODO. Per esempio, potresti voler avere il basico \texttt{TODO=/=DONE}, ma
anche un workflow per il bug fix. Il tuo setup potrebbe essere simile
a questo:

\begin{verbatim}
(setq org-todo-keywords
      '((sequence "TODO(t)" "|" "DONE(d)")
        (sequence "REPORT(r)" "BUG(b)" "KNOWNCAUSE(k)" "|" "FIXED(f)")))
\end{verbatim}

Le keyword dovrebbero essere tutte differenti, questo aiuta Org mode a
tenere traccia di quale sottosequenza debbe essere usata dall'oggetto
dato. L'esempio mostra, anche, come definire delle chiavi per un
accesso più veloce ad un particolare stato, aggiungendo una lettera
tra parentesi dopo ogni parola chiave - riceverai un prompt dopo aver
premuto C-c C-t.

Per definire delle keyword TODO che siano valide solo in un singolo
file, usa il seguente testo in una qualsiasi parte del file.

\begin{verbatim}
#+TODO: TODO(t) | DONE(d)
#+TODO: REPORT(r) BUG(b) KNOWNCAUSE(k) | FIXED(f)
#+TODO: | CANCELED(c)
\end{verbatim}

Dopo aver cambiato una di queste linee, usa C-c C-c con il
cursore nella linea dove hai fatto il cambiamento per farlo conoscere
alla modalità Org.

\subsection{Registrazioni dei progressi}
\label{sec:org642a354}
Per registrare un timestamp e una nota quando cambi lo stato di un
TODO, esegui il comando \texttt{org-todo} con un argomento prefisso.

\begin{description}
\item[{C-u C-c C-t (\texttt{org-todo})}] Fornisce un prompt per una nota e registra il tempo in cui é stato
cambiato lo stato del TODO.
\end{description}

Org mode può anche registrare automaticamente un timestamp e
opzionalmente una nota quando segni un elemento TODO come DONE, o ogni
volta che cambi lo stato di un elemento TODO. Questo sistema é
altamente configurabile, le impostazioni possono essere basate su una
parola chiave e possono essere localizzate in un file o ogni
sottoalbero. Per maggiori informazioni su come cronometrare il tempo
di lavoro vedi \ref{sec:orgc68efae}.

\subsubsection*{Chiudere un elemento}
\label{sec:org66339d3}
La più semplice registrazione é tenere traccia di \emph{quando} un certo
elemento TODO é segnato come completato. Questo può essere fatto
con\footnote{L'impostazione corrispondente nel buffer é \texttt{\#+STARTUP: logdone}.}

\begin{verbatim}
(setq org-log-done 'time)
\end{verbatim}

Nel momento in cui modifichi lo stato di un oggetto da TODO ( non
fatto ) in uno degli stati DONE, una linea \texttt{CLOSED: [timestamp]} viene
inserita subito sotto l'intestazione.

Se vuoi registrare una nota con il timestamp usa\footnote{L'impostazione corrispondente nel buffer é \texttt{\#+STARTUP:
logenotedone}.}

\begin{verbatim}
(setq org-log-done 'note)
\end{verbatim}

Ti verrà mostrato un prompt dove inserire la note e questa nota sarà
registrata sotto l'oggetto con un titolo \texttt{Closing Note}.

\subsubsection*{Tenere traccia degli stati TODO}
\label{sec:org76b4778}
Potresti voler tenere traccia dello stato del TODO. Puoi registrare
solo un timestamp o un time stamp con nota pre un cambiamento. Queste
registrazioni sono inserite dopo il titolo come elemeneti di
lista. Quando prendi una gran quantità di note, potresti volere tenere
le note esternamente in un drawer. Per questa funzionalità customizza
la variabile \texttt{org-log-into-drawer}.

Per la registrazioni dello stato, Org mode si basa sulla
configurazione delle parole chiave. Questo si ottiene aggiungendo un
marker speciale \textbf{!} ( per il timestamp ) e \textbf{@} ( per le note ) tra
parentesi dopo ogni parola chiave. Per esempio:

\begin{verbatim}
#+TODO: TODO(t) WAIT(w@/!) | DONE(d!) CANCELLED(c@)
\end{verbatim}


definedo le keyword TODO e le chiavi per l'accesso rapido, è anche
richiesto che un tempo sia registrato quando un oggetto viene settato
a \texttt{DONE}, e che una nota sia registrata quando si cambia a \texttt{WAIT} o
\texttt{CANCELED}. La stessa sintassi funziona anche quando setti
\texttt{org-todo-keywords}.

\subsection{Priorità}
\label{sec:org6fa8e75}
Se usi la modalità Org in maniera estesa, potresti trovarti con tanti
elementi TODO e quindi potrebbe aver senso dargli delle priorità. Si
possono dare priorità mettendo una \emph{priority cookie} nel titolo
dell'elemento TODO, tipo questo:

\begin{verbatim}
*** TODO [#A] Write letter to Sam Fortune
\end{verbatim}


Org mode supporta tre priorità \texttt{A}, \texttt{B} e \texttt{C}. \texttt{A} è la più alta, \texttt{B}
quella di default se non specificata. Le priorita creano una
differenza solo nell'agenda.

\begin{description}
\item[{C-c , (\texttt{org-priority})}] Imposta la priorità nel titolo corrente. Premi A,
B o C per selezionare la priorità o \texttt{SPC} per
rimuovere il cookie.

\item[{S-SU (\texttt{org-priority-up}); S-GIU (\texttt{org-priority-down})}] Aumentano/diminuiscono la piorità del titolo corrente.
\end{description}

\subsection[Breaking Down Tasks]{Dividi i compiti in sotto compiti}
\label{sec:org9d8d40b}
A volte é consigliabile dividere i grandi compiti in compiti più
piccoli, piccoli compiti gestibili. Puoi farlo creando una struttura
ad albero con sotto elementi TODO, con dettagliati i sottocompiti
dell'albero. Per avere un'anteprima della frazione dei sottocompiti
che hai già segnato come fatti, inserisci \texttt{[/]} o \texttt{[\%]} dove vuoi
nell'intestazione. Questi cookie saranno aggiornati ogni volta che lo
stato dei TODO figli cambia o quando premi C-c C-c nel
cookie. Per esempio:

\begin{verbatim}
* Organizzare un party [33%]
** TODO Chiamare le persone [1/2]
*** TODO Peter
*** DONE Sarah
** TODO Buy food
** DONE Parlare con il vicino
\end{verbatim}

\subsection{Liste selezionabili}
\label{sec:org62ce715}
Ogni elemento in una lista ( vedi \ref{sec:orgd18e700} ) può essere
trasformata in una casella di controllo ( checkbox ) che deve iniziare con
la stringa \texttt{[ ]}. Le checkbox non sono incluse nella lista di TODO
globale, sono molto comode per dividere un compito in un numero di
semplici step.

Qui c'é un esempio di una lista di checkbox.

\begin{verbatim}
* TODO Organizzare un party [2/4]
 - [-] Chiamare le persone
  - [ ] Peter
  - [X] Sarah
 - [X] Ordinare il cibo
\end{verbatim}

Le caselle di controllo lavorano in maniera gerarchica, quindi se un
elemento checkbox ha figli che sono checkbox, attivando una casella di
controllo figlia si fa in modo che nella casella di controllo padre
venga riflesso se nessuno, alcuni o tutti i figli sono stati selezionati.

I seguenti comandi operano con le checkbox:

\begin{description}
\item[{C-c C-c, C-u C-c C-c (\texttt{org-toggle-checkbox})}] Aggiorna lo stato di una checkbox o---con un argomento
prefisso---cambia la presenza della checkbox nel punto.

\item[{M-S-RET (\texttt{org-insert-todo-heading})}] Inserisce un nuovo elemento con una checkbox. Questo funziona solo
se ci si trova in un elemeto della lista ( vedi \ref{sec:orgd18e700} ).
\end{description}

\section{TAG}
\label{sec:org384d6f8}
Un modo eccellente per implementare le label ed i contesti, per le
informazioni incrociate correlate, é quello di assegare dei \emph{tag} alle
intestazioni. La modalità Org ha un supporto molto esteso per i tag.

Ogni intestazione puo contenere una lista di tag; sono messi alla fine
delle intestazioni. I tag sono normali parole che contangono lettere,
numeri, \texttt{\_} e \texttt{@}. I tag sono preceduti e succeduti da un singolo
duepunti, es. \texttt{:work:}. Alcuni tag possono essere specificati come in
\texttt{:work:urgent:}. I tag di default sono in grassetto dello stesso
colore dell'intestazione.

\subsection*{Ereditazione dei tag}
\label{sec:orgcc2d20c}
I tag usano la struttura gerarchica dell'albero di outline. Se
un'intesatzione ha un certo tag, tutti i sottotitoli ereditano il
tag. Per esempio nella lista

\begin{verbatim}
* Incontro con il gruppo Francese     :work:
** Sommario di Frank                  :boss:notes:
*** TODO Prepara le slide per lui     :action:
\end{verbatim}

l'intestazione finale ha i tag \texttt{work}, \texttt{boss}, \texttt{notes} ed \texttt{action} anche se nel titolo finale non vengono segnati
questi tag.

Puoi anche impostare i tag che tutte gli elementi nel file devono
ereditare semplicemente come se questi tag fossero definiti in un
ipotetico livello zero che sovrasta l'intero file. Usando una linea
del genere\footnote{Come tutto nelle impostazioni del buffer, premendo C-c C-c si attiva ogni cambiamento nella linea.}:

\begin{verbatim}
#+FILETAGS: :Peter:Boss:Secret:
\end{verbatim}

\subsection*{Impostare i taga}
\label{sec:org8d5182e}
I tag possono essere semplicemente inserite nel buffer alla fine di un
titolo. Dopo i duepunti, M-TAB offre il completamento dei
tag. Ci sono anche altri comandi speciali per inserire i tag:

\begin{description}
\item[{C-c C-q (\texttt{org-set-tags-command})}] Inserisce un nuovo tag nel titolo corrente. Org mode, inoltre, offre
il completamento o una speciale singola-chiave di interfaccia per
impostare i tag, vedi sotto.

\item[{C-c C-c (\texttt{org-set-tags-command})}] Quando il punto é in un'intestazione, questo compie la stessa azione di
C-c C-q.
\end{description}

Org supporta l'inserimento dei tag da una \emph{lista di tag}. Di default
questa lista é costruita dinamicamente, contiene tutti i tag usati, al
momento, nel buffer. Puoi anche specificare una lista fissa globale
con la variabile \texttt{org-tag-alist}. In fine puoi impostare i tag di
default da mettere in un file usando la keyword \texttt{TAGS}, tipo

\begin{verbatim}
#+TAGS: @work @home @tennisclub
#+TAGS: laptop car pc sailboat
\end{verbatim}


Di default Org mode usa la funzionalità standard di completamento del minibuffer per inserire i
tag. Comunque, implementa anche un'altro metodo di selezione, veloce, del tag chiamato \emph{fast tag
selection}. Questo ti permette di selezionare e deselezionare i tag con solo una presione di
una singola chiave. Per questo scopo é bene assegnare una singola lettera ai tuoi tag più
comunemente usati. Puoi fare questo globalmente configurando la variabile \texttt{org-tag-alist} nel tuo file
di init di Emacs. Per esempio, puoi trovare utile assegnare a degli elementi in diversi file con
\texttt{@home}. In questo caso puoi impostare qualcosa di simile:

\begin{verbatim}
(setq org-tag-alist '(("@work" . ?w) ("@home" . ?h) ("laptop" . ?l)))
\end{verbatim}

Se il tag é rilevante solo per il file su cui stai lavorando, allora
puoi inserire ed impostare la keyword \texttt{TAGS} come:

\begin{verbatim}
#+TAGS: @work(w)  @home(h)  @tennisclub(t)  laptop(l)  pc(p)
\end{verbatim}

\subsection*{Ragruppare i tag}
\label{sec:orgc7ee942}
Un tag puo essere definito come un \emph{gruppo di tag} per un insieme di
altri tag. Il gruppo di tag può essere visto come il ``termine più
ampio'' per un insieme di tag.

Puoi impostare un gruppo di tag usando le parentedi ed inserendo i
duepunti tra il gruppo di tag ed i tag relazionati:

\begin{verbatim}
#+TAGS: [ GTD : Control Presp ]
\end{verbatim}


o se i tag nel gruppo sono mutualmente esclusivi:

\begin{verbatim}
#+TAGS: { COntext : @Home @Work }
\end{verbatim}


Quando cerchi per un gruppo di tag, ti vengono ritornati tutti i membri del gruppo e i
sottogruppi. Nella visualizzazione agenda, filtrando per un gruppo di tag mostra o nasconde i titoli
segnati con almeno uno dei membri o del gruppo o ogniuno dei sottogruppi.

Se vuoi, temporaneamente, ingnorare i gruppi di tag, cambia il gruppo
dei dag supportati con \texttt{org-toggle-tags-groups}, mappato con
C-c C-x q.

\subsection*{Ricerca dei tag}
\label{sec:org156e7ff}
\begin{description}
\item[{C-c / m o C-c $\backslash$ (\textasciitilde{} org-match-sparse-tree\textasciitilde{})}] Crea un albero sparso con tutti i titoli che corrispondono ad un tag
cercato. Con un argomento prefisso C-u, ignora i titoli
che non sono una linea TODO.

\item[{M-x org-agenda m (\texttt{org-tags-views})}] Crea una lista glabale di tag corrispondenti, da tutti i file
dell'agenda. vedi \ref{sec:org17b9282}.

\item[{M-x org-agenda M (\texttt{org-tags-views})}] Crea una lista globale di tag corrispondenti, da tutti i file
dell'agenda, ma controlla solo gli elementi TODO.
\end{description}

Tutti questi comandi richiedono una stringa di corrispondenza che
rispetti la logia booleana di base \texttt{+boss+urgent-project1}, per
trovare gli elementi con i tag \texttt{boss} e \texttt{urgent}, ma non \texttt{project},
oppure \texttt{Kathy|Sally} per trovare un elemento che sia segnato come
\texttt{Kathy} o \texttt{Sally}. La sintassi completa per cercare queste stringhe é
ricca e permette anche la ricerca all'interno di keyword TODO, di
livelli e di proprieta'. Pe una più dettagliata descrizione con
diversi esempi, vedi \ref{sec:org17b9282}.

\section{Proprietà}
\label{sec:org306bfd2}
Le proprieta sono un associazione di chiave-valore associate ad un
oggetto. Risiedono in speciali drawer con nome \texttt{PROPERTIES}. Ogni
proprietà è specificata su una linea singola, con la chiave prima (
circondata dai duepunti ) ed il valore dopo.

\begin{verbatim}
* CD collection
** Classic
*** Goldberg Variations
    :PROPERTIES:
    :Title:     Goldberg Variations
    :Composer:  J.S. Bach
    :Publisher: Deutsche Grammophon
    :NDisks:    1
    :END:
\end{verbatim}

Potresti definire i valori permessi per una particolare proprietà
\texttt{Xyz} settando una proprietà \texttt{Xyz\_ALL}. Questa speciale proprietà è
\emph{ereditata}, quindi se lo imposti nell'oggetto di livello 1, viene
applicato agli oggetti dell'albero. Quando i valori permessi sono
definiti, l'impostazione della corrispondente proprietà diventa più
semplice e meno soggetta ad errori di battitura. Per l'esempio con la
collezione di CD, possiamo predefinire la casa discografica ed il
numero di dischi nella custodia tipo:

\begin{verbatim}
* CD collection
  :PROPERTIES:
  :NDisks_ALL:  1 2 3 4
  :Publisher_ALL: "Deutsche Grammophon" Philips EMI
  :END:
\end{verbatim}

Se vuoi impostare una proprietaà che può essere ereditata da ogni
oggetto nel file, usa una linea tipo:

\begin{verbatim}
#+PROPERTY: NDisks_ALL 1 2 3 4
\end{verbatim}


Il seguente comando aiuta a lavorare con le proprietà:

\begin{description}
\item[{C-c C-x p (\texttt{org-set-property})}] Imposta una proprietà. Questo richiede un nome ed un valore.

\item[{C-c C-c d (\texttt{org-delete-property})}] Rimuove una proprietà dall'oggetto corrente.
\end{description}

Per creare un albero sparso e liste speciali con la selezione basata
sulle proprietà, gli stessi comandi sono usati per cercare tag ( vedi
[BROKEN LINK: *Tag] ) . La sintassi per la stringa di ricerca è descritta nella
sezione \ref{sec:org17b9282}.

\section{Data e orario}
\label{sec:org733fa6d}
Per assistere la pianificazione, gli elementi TODO possono essere
etichettati con una data e/o un orario. La speciale stringa formattata
trasporta le informazioni di data ed orario chiamata, in Org mode,
timestamp. Questo può creare un pò di confusione in quanto il
timestamp é a volte usato per indicare quando qualcosa é stato creato
o l'ultimo cambiamento. Comunque, nella modalità Org questo termine
viene usato in questo senso più ampio.

Timestamp può essere usato per pianificare appuntamenti, schedulare
impegni, impostare scadenze, tracciare il tempo ed altro. La sezione
seguente descrive il formato timestamp e gli strumenti che Org mode ci
mette a disposizione per i casi d'uso che riguardano il tempo e gli
intervalli di tempo.

\subsection{Timestamp}
\label{sec:orgffbfc92}
Un timestamp é una specifica data---possibilemente con un orario o un
range di orari---in uno speciale formato, come \texttt{<2003-09-16 Tue>} o
\texttt{<2003-09-16 Tue 09:39>} oppure \texttt{<2003-09-16 Tue 12:00-12:30>}.
Un timestamp puo essere inserito ovunque nelle intestazioni o nel corpo di un
elemento dell'albero Org. La sua presenza genera la visualizzazione di
un elemento in una specifica data nell'agenda ( vedi [BROKEN LINK: L'aggenda Settimanale/Giornaliera] ). Distinguiamoli:

\begin{description}
\item[{Timestamp semplici; Eventi; Appuntamenti}] Un semplice timestamp che assegna ad un elemeto una
data/orario. Questo é come scrivere un appuntamento o un evento in
un'agenda di carta.

\begin{verbatim}
* Incontro con Peter al cinema
  <2006-11-01 Wd 19:15>
* Discussione sui cambiamenti climatici
  <2006-11-02 Thu 20:00-22:00>
* Giorni di ferie
  <2006-11-03 Fri>
  <2006-11-06 Mon>
\end{verbatim}

\item[{Timestamp con un intervallo che si ripete}] Un timestamp contenente un \emph{intervallo di ripetizione}, indica
che, lo stesso, non si ripeta solo una volta ma ancora e ancorra
dopo un certo intervallo di N ore (h), giorni (d), settimane (w),
mesi (m) o anni (y). Di seguito ne mostriamo uno che si ripete ogni
Marcoledì:

\begin{verbatim}
* Prendere Sam da scuola
  <2007-05-16 Wed 12:30 +1w>
\end{verbatim}

\item[{Espressioni per gli elementi in stile diario}] Per specificare diverse date complesse, Org mode supporta l'uso di
speciali espressioni di elementi diario, implementati nel pacchetto
Emacs Calendar. Per esempio, con un tempo opzionale

\begin{verbatim}
* 22:00-23-00 L'incontro dei nerd ogni secondo Giovedi del mese
  <%%(diary-float t 4 2)>
\end{verbatim}

\item[{Range di tempo}] Un range é un timestamp connesso da due \texttt{-}.

\begin{verbatim}
* Discussione sui cambiamenti climatici
  <2006-11-02 Thu 10:00-12:00>
\end{verbatim}

\item[{Range di Tempo/Data}] Due timestamp connessi da \texttt{-{}-{}} denotano un range. Nell'agenda,
l'intestazione é mostrata nella prima e nell'ultima data del range,
e in ogni data che rientra nel range. Il primo esempio é specificato
soltanto le data del range invece nel secondo esempio viene
specificato un intervallo di tempo per entrambe le date.

\begin{verbatim}
** Incontro ad Amsterdam
   <2004-08-23 Mon>--<2004-08-26 Thu>
** Riunioni del comitato di questa settimana
   <2004-08-23 Mon 10:00-11:00>--<2004-08-26 Thu 10:00-11:00>
\end{verbatim}

\item[{Timestamp inattivo}] Come un semplice timestamp, ma con le parentesi quadre al posto
delle angolari. Questo timestamp sarà inattivo nel senso che \emph{non}
innescherà un elemento da mostrare nell'agenda.

\begin{verbatim}
* Gillian arriva tardi per la quinta volta
  [2006-11-01 Wed]
\end{verbatim}
\end{description}

\subsection{Creare un timestamp}
\label{sec:org816e6ea}
Per fare in modo che Org mode possa riconoscere i timestamp devono
essere in uno specifico formato. Tutti i comandi descritti sotto
producono un timestamp nella modalità corretta.

\begin{description}
\item[{C-c . (\texttt{org-timestamp})}] Richiede una date ed inserisce un corrispondete timestamp. Quando il
punto é in un timestamp pre esistente nel buffer, il comando viene
usato per modificare questo timestamp invece di inserine uno
nuovo. Quando questo comando viene usato due volte in successione un
range di tempo viene inserito. Con un argomento prefisso aggiunge
anche l'orario corrente.

\item[{C-c ! (\texttt{org-timestamp-inactive})}] Come C-c ., ma inserisce un timestamp inattivo che non
genera un elemento agenda.

\item[{S-SINISTRA (\texttt{org-timestamp-down-day}), S-DESTRA (\texttt{org-timestamp-up-day})}] Cambia la data, di un giorno, nel punto corrente.

\item[{S-SU (\texttt{org-timestamp-up}), S-GIU (\texttt{org-timestamp-down})}] All'inizio o sulle parentesi graffe di un timestamp, ne cambia il
tipo. All'interno di un timestamp ne modifica l'elemento al
punto. Il punto può essere un anno, un mese, un giorno, un'ora o un
minuto. Quando il timestamp contiene un intervallo di tempo tipo
\texttt{15:30-16:30}, modificando il primo verrà modificato anche il
secondo, spostando, il blocco di tempo, di lunghezza costante. Per
cambiare la lunghezza si deve modificare il secondo orario.
\end{description}

Quando Org mode richiede una data/orario, accetta qualsiasi stringa
contenente delle informazione su data e/o orario e,
intelligentemente, interpreta la stringa, ricavando valori predefiniti
per le informazioni non specificate dalla data e ora correnti. Puoi
anche selezionare una data nel calendario pop-up. Consulta il \href{org}{manuale}
per ulteriori informazioni su come funziona esattamente la richiesta
di data/orario.

\subsection{Scadenze e schedulazioni}
\label{sec:org0e40b68}
Un timestamp potrebbe essere preceduto da delle speciali keyword per
facilitarne la pianificazione:

\begin{description}
\item[{C-c C-d (\texttt{org-deadline})}] Inserisce la parola chiave \texttt{DEADLINE} assieme al timestamp, nella
linea dopo l'intestazione.

Significato: l'attività---molto probabilemte una voce TODO, anche se
non necessariamente---si suppone sia stata completata in quella
data.

Nella data di scadenza, il task é mostrato nell'agenda. Inoltre,
l'agenda per \emph{oggi} mostra un avviso riguardante l'approssimarsi o
la mancata scadenza, a paritre da \texttt{org-deadline-warning-days} prima
della data di scadenza e continuando fino a che l'elemento non sarà
segnato come fatto. Un esempio:

\begin{verbatim}
*** TODO Scrivere un articolo sulla Terra per la Guida
    DEADLINE: <2004-02-29 Sun>
    L'editore in carica e' [[bbdb:Ford Prefect]]
\end{verbatim}

\item[{C-c C-s (\texttt{org-schedule})}] Inserisce la parola \texttt{SCHEDULED} con il timestamp sotto la linea
dell'intestazione.

Significato: Hai pianificato di iniziare quel lavoro in quella data\footnote{Questo é completamente diverso da ciò che normalmente si intende
\emph{pianificare un appuntamento}, che in Org viene eseguita semplicemente
inserendo un timestamp senza keyword.}

Il titolo é mostrato nella data\footnote{Sarà comunque mostrato in quella data anche se verrà
contrassegnato come fatto. Se non ti piace, imposta la variabile
\texttt{org-agenda-skip-scheduled-if-done}.} impostata. In aggiunta, nella
compilazione per \emph{oggi}, é presente un promemoria che la data
pianificata é trascorsa, fino a che l'elemento non sarà segnato come
fatto, ovvero l'attività viene automaticamente inoltrata fino al
completamento.

\begin{verbatim}
*** TODO Chiamare Trillian per un appuntamento a New Years Eve.
    SCHEDULED: <2004-12-25 Sat>
\end{verbatim}
\end{description}

Spesso i compiti hanno la necessità di essere ripetuti ancora ed
ancora. Org mode aiuta nell'organizzare di questi lavori usando un, così
detto, repeater in una \texttt{DEADLINE}, \texttt{SCHEDULED} o in un timestamp
semplice. Nel seguente esempio:

\begin{verbatim}
** TODO Paga l'affitto
   DEADLINE: <2005-10-01 Sat +1m>
\end{verbatim}

il \texttt{+1m} é un repeater; l'interpretazione che si intende é che il
lavoro ha una data di scadenza nel \texttt{<2005-10-01>} e si ripete ogni (
un ) mese partendo da quella data.

\subsection{Cronometra il tempo di lavoro}
\label{sec:orgc68efae}
La modalità Org ti permette di cronometrare quanto tempo ti serve per
uno specifico proggetto.

\begin{description}
\item[{C-c C-x C-i (\texttt{org-clock-in})}] Fa partire il cronometro nell'elemento corrente (clock-in). Questo
inserisce la parola chiave \texttt{CLOCK} assieme ad un timestamp. Quando
viene richiamato assimen ad un argomento prefisso, seleziona il task
da una lista di task chiusi di recente.

\item[{C-c C-x C-o (\texttt{org-clock-out})}] Ferma il timer (clock-out). Questo inserisce un'ulteriore timestamp
nella stessa posizione dove il cronometro aveva inserito l'ultimo
inizio. Viene anche direttamente calcolato ed inserito il tempo dopo
il range di tempo nella forma \texttt{=>HH:MM}.

\item[{C-c C-x C-e (\texttt{org-clock-modify-effort-estimate})}] Aggiorna lo sforzo richiesto per il corrente timer di lavoro.

\item[{C-c C-x C-q (\texttt{org-clock-cancel})}] Cancella il cronometro corrente. Questo é utile se il cronometro é
partito per sbaglio o se finisci di lavorare o altro.

\item[{C-c C-x C-j (\texttt{org-clock-goto})}] Salta al titolo del lavoro del corrente timer in lavoro. Con un argomento
prefisso C-u seleziona il lavoro targhet da una lista di lavori
cronometrati di recente.
\end{description}

La chiave l viene solitamente usata nell'agenda ( vedi
[BROKEN LINK: *L'aggenda Settimanale/Giornaliera] ), mostra su quali task stai
lavorando o quelli che hai chiuso durante il giorno.

\section{Cattura, sposta e archivia}
\label{sec:org15f3948}
Una parte importante di ogni sistema di organizzazione é l'abilità di
catturare rapidamente nuove idee e task, e associare il materiale di
riferimento con esse. Org lo fa usando un processo detto
\emph{capture}. Può anche archiviare i relativi file di un lavoro (
\emph{allegati} ) in una speciale directory. Una volta nel sistema, le
attività ed i processi devono essere spostati. Muovere completamente
un albero di proggetto in un archivio di file rende il sistema
compatto e veloce.

\subsection{Cattura}
\label{sec:org1358116}
Capture ti permette di archiviare velocemente le note con una piccola
interruzione del tuo processo di lavoro. Puoi definire modelli per i
nuovi elementi ed associarli con differenti target per salvare le
note.

\subsubsection*{Impostare capture}
\label{sec:org8936a99}
L'impostazione sequente setta un file target\footnote{Usando i modelli di capture, puoi avere un controllo più preciso
sulla posizione della cattura. vedi \hyperref[sec:org14346c5]{Modelli di cattura}} di default per le
note.

\begin{verbatim}
(setq org-default-notes-file (concat org-directory "/notes.org"))
\end{verbatim}

Puoi anche definire una combinazione globale per catturare nuovo
materiale ( vedi \hyperref[sec:org316fc0b]{Attivazione} ).

\subsubsection*{Usare capture}
\label{sec:org0c11add}
\begin{description}
\item[{m-x org-capture (\texttt{org-capture})}] Inizia il processo di cattura, ti posiziona in un buffer limitato
per la modifica.

\item[{C-c C-c (\texttt{org-capture-finalize})}] Uno volta che hai finito di inserire le informazioni nel buffer di
cattura, C-c C-c ti riporta nella finestra da dove hai
fatto partire il processo, in modo che tu possa riprendere il lavoro
senza altre distrazioni.

\item[{C-c C-w (\texttt{org-capture-refile})}] Finalizza il processo di cattura mettendo le note in una
differente posizione ( vedi [BROKEN LINK: Soposta e Copia] ).

\item[{C-c C-k (\texttt{org-capture-kill})}] Annulla il processo di cattura e ti riporta allo stato precedente.
\end{description}

\subsubsection*{Modelli di cattura}
\label{sec:org14346c5}
Puoi usare i modelli per differenti tipi di oggetti di cattura, e per
differenti posizioni. Per dire se vuoi usare un template per creare
degli oggetti TODO generali, e se vuoi mettere questi elementi sotto
un intestazione \texttt{Tasks} nel tuo file \texttt{\textasciitilde{}/org/gtd.org}. Puoi anche
catturare degli oggetti giornalieri come un albero di date nel file
\texttt{journal.org}. Una possibile configurazione protrebbe essere simile a
questa:

\begin{verbatim}
(setq org-capture-templates
      '(("t" "Todo" entry (file+headline "~/org/gtd.org" "Tasks")
         "* TODO %?\n  %i\n  %a")
        ("j" "Journal" entry (file+datetree "~/org/journal.org")
         "* %?\nEntered on %U\n  %i\n  %a")))
\end{verbatim}

Se premi t quando sei nel menu di capture, Org preparerà il
template per te in questo modo:

\begin{verbatim}
* TODO
  [[file:COLLEGAMENTO DI DOVE HAI INIZIATO LA CATTURA]]
\end{verbatim}


Durante l'espansione del template, il carattere speciale di escape
\% \footnote{Se hai bisogno di una di queste sequenze letterali, usa il
carattere di escape \texttt{\textbackslash{}} con \texttt{\%}.} permette l'inserimento dinamico del contenuto. Qui c'é una
piccola selezione di possibilità, consulta il \href{manual}{manuale} per averne di
più.

\begin{center}
\begin{tabular}{ll}
\texttt{\%a} & annotazione, normalmente il link creato con \texttt{org-store-link}\\[0pt]
\texttt{\%i} & contenuto iniziale, la regione dove capture é chiamata con C-u\\[0pt]
\texttt{\%t \%T\%} & timestamp, solo data o data ed orario.\\[0pt]
\texttt{\%u \%U} & come sopra ma per i timestamp inattivi\\[0pt]
\texttt{\%?} & dopo avere completato il template, posiziona il punto li.\\[0pt]
\end{tabular}
\end{center}

\subsection{Sposta e Copia}
\label{sec:org800e315}
Quando revisioni i dati catturati, potresti volere spostare o copiare
alcuni dei tuoi elementi in una lista differente, per esempio in un
progetto. Tagliare, cercare la posizione corretta e poi incollare la
nota é scomodo. Per semplificare il processo puoi usare gli speciali
comandi di seguito:

\begin{description}
\item[{C-c C-w (\texttt{org-agenda-refile})}] Sposta gli oggetti o le regioni in un punto. Questo comando offre
possibili posizioni dove spostare gli oggetti e ti permette di
selezionarne uno con il completamento. L'oggetto ( o tutti gli
oggetti nella regione ) é inserito sotto l'intestazione target come
sotto oggetti.

Di default, tutti i livelli 1 delle intestazioni nel buffer corrente
sono considerati come target, ma puoi avere delle definizioni molto
più complesse riguardo un certo numero di file. Vedi la variabile
\texttt{org-refile-targets} per dettagli.

\item[{C-c C-c C-w (\texttt{org-agenda-refile})}] Usa l'interfaccia di spostameto per saltare ad un titolo.

\item[{C-c Cu C-c C-w (\texttt{org-refile-goto-last-stored})}] Salta ad una posizione dove \texttt{org-refile} ha spostato l'albero
l'ultima volta.

\item[{C-c M-w (\texttt{org-refile-copy})}] La copia funziona come lo spostamento, fatta eccezione che le note
originali non vengono cancellate.
\end{description}

\subsection{Archiviare}
\label{sec:org892e25f}
Quando un progetto rappresentato da un (sotto)albero é concluso,
potresti volerlo spostare fuori dai piedi e interrompere i suoi
contributi all'agenda. L'archiviazione é importante per tenere i tuoi
file di lavoro compatti e la vista della ricerca globale della'agenda
più veloce.

La più comune azione di archiviazione é di spostare un albero di
progetto in un altro file, il file di archivio.

\begin{description}
\item[{C-c C-x C-a (\texttt{org-archive-subtree-default})}] Archivia il corrente elemento usando un comando specificato nella
variabile \texttt{org-archive-default-command}.

\item[{C-c C-x C-s o brevemente C-c \$ (\texttt{org-archive-subtree})}] Archivia il sottoalbero partendo dalla posizione del punto nella
posizione data da \texttt{org-archive-location}.
\end{description}

La posizione di archiviazione standard é un file nella stessa
directory del file corrente, con il nome derivato aggiungendo
\texttt{\_archive} al nome del file corrente. Puoi anche scegliere sotto quale
intestazione archiviare gli elementi, con la possibilità di
aggiungerli ad un albero delle date in un file. Per informazioni ed
esempi su come specificare il file e le intestazioni, vedi la documentazione
sulla stringa della variabile \texttt{org-archive-location}.

C'é anche un'opzione nel buffer per settare questa variabile, per
esempio:

\begin{verbatim}
#+ARCHIVE: %s_done::
\end{verbatim}

\section{Visualizzazioni Agenda}
\label{sec:org3ad7d0f}
A causa del modo in cui lavora Org, gli elementi TODO, gli elementi di
timestamp ed i titoli taggati possono essere ovunque nel file o in
file diversi. Per avere un visione delle azioni in corso o degli
eventi che sono importanti per una data particolare, queste
informazioni devono essere raccolte, ordinate e visualizzate in modo
organizzato.

Le informazioni estratte sono mostrate in uno speciale \emph{buffer
agenda}. Questo buffer é in sola lettura, ma fornisce comandi per
visitare le corrispondenti posizioni nei file Org originali e anche
modificare questi file da remoto. La modifica remota dal buffer agenda
significa, per esempio, che puoi cambiare le date di una scadenza e
gli appuntamenti da quel buffer agenda. Per una lista di comandi
avviabili nel buffer Agenda, vedi \ref{sec:org15525c6}.

\subsection{File agenda}
\label{sec:orgb467595}
L'informazione mostrata è normalmente raccolta da tutti i \emph{file
agenda}, i file sono listati nella variabile \texttt{org-agenda-files}.

\begin{description}
\item[{C-c [ (\texttt{org-agenda-file-to-front})}] Aggiunge il file corrente nella lista dei file agenda. Il file è
aggiunto all'inizio della lista dei file. Se si trova già nella
lista, viene spostato all'inizio. Con un argomento prefisso, il file
viene aggiunto/spostato alla fine.

\item[{C-c ] (\texttt{org-remove-file})}] Rimuove il corrente file dalla lista dei file agenda.

\item[{C-' / C-$\backslash$ (\texttt{org-cycle-agenda-files})}] Cicla tra la lista dei file agenda, visita un file dopo l'altro.
\end{description}

\subsection[Agenda Dispatcher]{Il Distributore Agenda}
\label{sec:orga0feddf}
Le viste sono create attraverso un distributore, accessibile con
M-x org-agenda, o, meglio abbinato ad una chiave globale (
vedi \hyperref[sec:org316fc0b]{Attivazione} ). Mostra un menu in cui una lettera aggiuntiva
viene richiesta per eseguire il comando. Il distributore offre i
seguenti comandi predefiniti:

\begin{description}
\item[{a}] Crea un agenda tipo calendario ( vedi \ref{sec:orgdc310a7} ).

\item[{t, T}] Crea una lista di tutti gli elementi TODO ( vedi \ref{sec:org0f74436}
).

\item[{m, M}] Crea una lista di titoli corrispondenti all'espressione data ( vedi
\ref{sec:org17b9282} ).

\item[{s}] Crea una lista di oggetti selezionati tramite un espressione
booleana di keyword e/o espressioni regolari che deve o non deve
verificarsi nelgli oggetti.
\end{description}

\subsection[Built-in Agenda Views]{L'agenda Settimanale/Giornaliera}
\label{sec:orgdc310a7}
Il proposito dell'\emph{agenda} settimanale/giornaliera é di funzionare
come una pagina di un agenda cartacea, mostrando tutti i lavori per la
settimana od il giorno corrente.

\begin{description}
\item[{M-x org-agenda a (\texttt{org-agenda-list})}] Compila un agenda per la settimana corrente da una lista di file
Org. L'agenda mostra gli oggetti per ogni giorno.
\end{description}

Org mode capisce la sintassi del diario e ti permette di usare degli
elementi espressione agenda direttamente nei file Org:

\begin{verbatim}
* Holidays
  :PROPERTIES:
  :CATEGORY: Holiday
  :END:
%%(org-calendar-holiday)   ; Funzione speciale per il nome holiday

* Birthdays
  :PROPERTIES:
  :CATEGORY: Ann
  :END:
%%(org-anniversary 1956  5 14) Arthur Dent is %d years old
%%(org-anniversary 1869 10  2) Mahatma Gandhi would be %d years old
\end{verbatim}

Org può interagire con la funzione di nodifica degli appuntamenti di
Emacs. Per aggiungere un appuntamento ai tuoi file agneda, usa il
comando \texttt{org-agenda-to-appt}.

\subsection[Global TODO List]{Lista globale TODO}
\label{sec:org0f74436}
La lista globale dei TODO contiene tutti gli oggetti TODO, non
completati, formattati e raggruppati in un singolo punto. La modifica
remota degli elementi TODO ti da la possibilità di cambiare lo stato
di un oggetto TODO con una singola pressione di un tasto. Per i
comandi avviabili nella TODO list, vedi \ref{sec:org15525c6}.

\begin{description}
\item[{M-x org-agenda t (\texttt{org-todo-list})}] Mostra la lista globale dei TODO. Questo ragruppa gli elementi TODO
da tutti i file agenda ( vedi \ref{sec:org3ad7d0f} ) in un unico
buffer.

\item[{M-x org-agenda T (\texttt{org-todo-list})}] Come sopra, ma permette la selezione di uno specifica keyword TODO.
\end{description}

\subsection{Trova i tag e le proprietà}
\label{sec:org17b9282}
Se le intestazioni nel file agenda sono marcati con i \emph{tags} ( vedi
[BROKEN LINK: *Tag] ), o ha delle proprietà ( vedi \ref{sec:org306bfd2} ), puoi selezionare le
intestazioni in base a questi metadati e raccoglierli in un buffer
agenda. La stessa cosa qui descittta é applicabile anche quando crei
gli alberi sparsi con C-c / m.

\begin{description}
\item[{M-x org-agenda m (\texttt{org-tags-views})}] Produce una lista di tutte le intestazioni che trova con un dato set
di tag. Il comando richiede l'inserimento di un criterio di
selezione, che é un espressione logica booleana con tag, come
\texttt{+work+urgent-witboss} o \texttt{work|home} ( vedi [BROKEN LINK: *Tag] ). Se usi spesso
una specifica ricerca, puoi definire un comando personalizzato (
vedi \ref{sec:orga0feddf} ).

\item[{M-x org-agenda M(\texttt{org-tags-view})}] Tipo m ma seleziona solo i titoli che sono oggetti TODO.
\end{description}

Una stringa di ricerca puo usare operatori Booleani \texttt{\&} per AND e \texttt{|}
per OR. \texttt{\&} si lega più fortemente di \texttt{|}. Le parentesi non sono
ancora implementate. Ogni elemento nella ricerca ha o un tag,
un'espressione regolare che trova un tag o un espressione tipo
\texttt{PROPERTY OPERATOR VALUE} con un operatore di comparazione, accede ad
una valore proprietà. Ogni elemento dev'essere preceduto da \texttt{-} per
non selezionarlo e un \texttt{+} che é zucchero sintattico per la selezione
positiva. L'operatore AND \texttt{\&} é opzionale quando \texttt{+} o \texttt{-} sono
presenti. Eccco alcuni esempi, usando solo i tag.

\begin{description}
\item[{\texttt{+work-boss}}] Seleziona i titoli taggati con \texttt{work}, ma scarta quelli che sono
marcati \texttt{boss}

\item[{\texttt{work|laptop}}] Seleziona le linee taggate \texttt{work} o \texttt{laptop}.

\item[{\texttt{work|laptop+night}}] Come sopra ma richiede che le linee \texttt{laptop} siano anche segnate
come \texttt{nihgt}.
\end{description}

Puoi anche cercare le proprietà e i tag nello stesso tempo, vedi il
\href{manul}{manuale} per informazinoi aggiuntive.

\subsection{Vista di ricerca}
\label{sec:orgb24d7ca}
Questa vista agenda é una funzione di ricerca generale del testo per
gli oggetti Org mode. É particolamente utile per trovare le note.

\begin{description}
\item[{M-x org-agenda s (\texttt{org-search-view})}] Questo é una speciale ricerca che permette di selezionare gli
elementi che corrispondono ad una sottostringa o a specifiche voci
usando la logica booleana.
\end{description}

Per esempio, la stringa di ricerca \texttt{computer equipment} trofva gli
elementi che contengono \texttt{computer equipment} come una sotto stringa.

La vista di ricerca può anche ricercare per specifiche keyword negli
oggetti, usando la logica booleana. La stringa di ricerca \texttt{+computer
+wifi -ethernet -\{8`.11[bg]\}} trova le note che tengono le parole
chiave \texttt{computer} e \texttt{wifi} ma non la keyword \texttt{ethernet} e che non
corrispondano all'espressione regolare \texttt{8\textbackslash{}.11[bg]}. che esclude sia
\texttt{8.11b} che \texttt{8.11g}.

Nota che in aggiunta ai file agenda, questo comando cerca anche nei
file elencati in \texttt{org-agenda-text-search-extra-files}.

\subsection[Agenda Commands]{Comandi nel Buffer Agenda}
\label{sec:org15525c6}
Gli oggetti nel buffer agenda sono un rimando ai file Org o ai file
del diario da dove vengono generati. Non ti é permesso modificare lo
stesso buffer agenda, ma i comandi ti permettono di vedere e di
saltare agli oggetti nella posizione originale e modificare i file Org
originarli da ``remoto'' partendo dal buffer agenda. Questa é solo una
selezione dei principali comandi, esplora il menu agenda ed il \href{manual}{manuale}
per la lista completa.

\subsubsection*{Movimento}
\label{sec:orgbbfd5f2}
\begin{description}
\item[{n (\texttt{org-agenda-next-line})}] Linea succesiva ( come GIU e C-n ).

\item[{p (\texttt{org-agenda-previous-line})}] Linea precedente ( come SU e C-p ).
\end{description}

\subsubsection*{Vedere/Andare ad un file Org}
\label{sec:orgc00560c}
\begin{description}
\item[{SPC (\texttt{org-agenda-show-and-scroll-up})}] Mostra la posizione originale dell'elemento in un'altra
finestra. Con un argomento prefisso, si assicura che i drawers
rimangano chiusi.

\item[{TAB (\texttt{org-agenda-goto})}] Va alla posizione originale dell'oggetto in un'altra finestra.

\item[{RET (\texttt{org-agenda-switch-to})}] Va alla posizione originale dell'oggetto e chiude le altre finestre.
\end{description}

\subsubsection*{Cambia visualizzazione}
\label{sec:org1ca9781}
\begin{description}
\item[{o (\texttt{delete-other-windows})}] Elimina le altre finestre.

\item[{v d o brevemente d (\texttt{org-agenda-day-view})}] Passa alla vista del giorno

\item[{v w o brevemente w (\texttt{org-agenda-week-view})}] Passa alla vista della settimana

\item[{f (\texttt{org-agenda-later})}] Va avanti nel tempo per visualizzare l'intervallo successivo a
quello corrente. Per esempio, se la visualizzazione copre la
settimana, passa alla settimana seguente.

\item[{b (\texttt{org-agenda-earlier})}] Torna indietro nel tempo per visualizzare le date precedenti.

\item[{. (\texttt{org-agenda-goto-today})}] Va ad oggi.

\item[{j (\texttt{org-agenda-goto-date})}] Richiede una data e ci va.

\item[{v l or v L o brevemente l (\texttt{org-agenda-log-mode})}] Attiva/disattiva la modalità Logbook. Nella modalità Logbook gli
oggetti che sono segnati come completati, mentre era attiva la
registrazione ( vedi la variabile \texttt{org-log-done} ), sono mostrati
nell'agenda, così come le voci che sono state cronometrate in quel
giorno. Quando chiamato con un argomento prefisso C-u,
mostra tutti i possibili oggetti logbook, includendo i cambi di
stato.

\item[{r, g (\texttt{org-agenda-redo})}] Ricarica il buffer agenda, per esempio per aggiornare i cambiamenti
dopo una modifica del timestamp degli items.

\item[{s (\texttt{org-save-all-org-buffers})}] Salva tutti i buffer Org nella sessione di Emacs corrente e anche la
posizione degli ID.
\end{description}

\subsubsection*{Modifica remota}
\label{sec:orga5c6d00}
\begin{description}
\item[{0--9}] Agomento numerico

\item[{t (\texttt{org-agenda-todo})}] Cambia lo stato TODO dell'elemento, sia nell'agenda che nel file di
origine.

\item[{C-k (\texttt{org-agenda-kill})}] Cancella l'oggetto corrente dall'agenda e lungo l'intero sottoalbero
nell'file Org originale.

\item[{C-c C-w (\texttt{org-agenda-refile})}] Muove l'oggetto al punto.

\item[{a (\texttt{org-agenda-archive-default-with-confirmation})}] Archivia il sottoalbero corrispondente all'elemento del punto usando
il comando di archiviazione di default impostato con nella variabile
\texttt{org-archive-default-command}.

\item[{\$ (\texttt{org-adenda-archive})}] Archivia il sottoalbero corrispondente all'intestazione corrente.

\item[{C-c C-s (\texttt{org-agenda-schedule})}] Schedula questo elemento. Con un argomento prefisso, rimuove il
timestamp schedulato.

\item[{C-c C-d (\texttt{org-agenda-deadline})}] Imposta una scadenza per questo elemento. Con un argomento prefisso,
rimuove la scadenza.

\item[{S-DESTRA (\texttt{org-agenda-do-date-later})}] Cambia il timestamp associato alla linea corrente con un giorno nel
futuro.

\item[{S-SINISTRA (\texttt{org-agenda-do-date-earlier})}] Cambia il timestamp associato alla linea corrente con un giorno nel
passato.

\item[{I (\texttt{org-agenda-clock-in})}] Inizia il cronometro nell'elemento corrente.

\item[{O (\texttt{org-agenda-clock-out})}] Ferma il cronometro precedentemente attivato.

\item[{X (\texttt{org-agenda-clock-cancel})}] Cancella il cronometro attivo.

\item[{J (\texttt{org-agenda-clock-goto})}] Va all'orologio corrente in un'altra finestra.
\end{description}

\subsubsection*{Chiudi ed esci}
\label{sec:org743c9e3}
\begin{description}
\item[{q (\texttt{org-agenda-quit})}] Chiude l'agenda, rimuove il buffer agenda

\item[{x (\texttt{org-agenda-ecit})}] Esce dall'agenda, rimuove il buffer agenda e tutti i buffer caricati
da Emacs per la compilazione dell'agenda.
\end{description}

\subsection{Personalizzare la vista agenda}
\label{sec:org9401c07}
La prima applicazione di personalizzazione di ricerca é la definizione
di una combinazione per la ricerca usata più frequentemente, o creando
un ordine del giorno oppure uno sparse tree ( la lettera copre soltanto
il buffer corrente).

I comandi personalizzati sono configurati nella variabile
\texttt{org-agenda-custom-commands}. Puoi personalizzare questa variabile,
per esempio premendo C dal dispacher agenda ( vedi \ref{sec:orga0feddf} ). Poui anche impostarla direttamente con Emacs
Lisp nel file di init di Emacs. Il seguente esempio continete tutte
viste valide dell'agenda.

\begin{verbatim}
(setq org-agenda-custom-commands
      '(("w" todo "WAITING")
        ("u" tags "+boss-urgent")
        ("v" tags-todo "+boss-urgent")))
\end{verbatim}

La stringa iniziale in ogni elemento definisce le chiavi che puoi
premere dopo l'avvio del comando del dispacher per poter accedere al
comando. Solitamente questo é di un solo carattere. Il secondo
parametro é il tipo di ricerca, seguito dalla stringa o espressione
regolare da usare per la selezione. L'esempio sopra quindi definirà:

\begin{description}
\item[{w}] Come per la ricerca globale per gli oggetti TODO con \texttt{WAITING} come
parola chiave del TODO

\item[{u}] Come per la ricerca globale di tags per i titoli segnati \texttt{boss} ma
non \texttt{urgent}.

\item[{v}] La stessa ricerca, ma limitata per i titoli che sono anche elementi
TODO.
\end{description}

\section[Markup]{Markup per contenuti ricchi}
\label{sec:org3563413}
Org sopratutto riguarda l'organizza e la ricerca all'interno delle tue
note di testo semplice. Comunque, fornisce un leggero ma robusto
linguaggio di markup per la foramattazione del testo ed altro. Usato
assieme al framework di espostazione ( vedi \ref{sec:orgc8db748} ), con Org
puoi produrre documenti molto belli.

\subsection{Paragrafi}
\label{sec:orgb9f47e4}
I paragrafi sono separati tra loro da una linea vuota. Se hai bisogno
di forzare un'interruzione di linea all'interno di un paragrafo, usa
\texttt{\textbackslash{}\textbackslash{}} alla fine della linea.

Per preservare l'interruzione di linea, indentazione e linee bianche
in una regione, ma anche usare la formattazione normale, puoi usare
questo costrutto, puoi anche usarlo per formattare la poesia.

\begin{verbatim}
#+BEGIN_VERSE
 Great clouds overhead
 Tiny black birds rise and fall
 Snow covers Emacs

    ---AlexSchroeder
#+END_VERSE
\end{verbatim}

Quando quoti un passaggio da un'altro documento, é costume formattarlo
come un paragrafo che é indentato in entrambi i margini sinistro e
destro. Puoi mettere questa citazione in un documento Org nel modo
seguente:

\begin{verbatim}
#+BEGIN_QUOTE
Everything should be made as simple as possible,
but not any simpler ---Albert Einstein
#+END_QUOTE
\end{verbatim}

Se vuoi centrare del testo, devi fare in questo modo:

\begin{verbatim}
#+BEGIN_CENTER
Everything should be made as simple as possible, \\
but not any simpler
#+END_CENTER
\end{verbatim}

\subsection{Enfasi e monospace}
\label{sec:org29c1023}
Puoi creare parole \texttt{*bold*}, \texttt{/italic/}, \texttt{\_sottolineate\_},
\texttt{=verbatim=} e \texttt{\textasciitilde{}codice\textasciitilde{}}, e, se non basta, \texttt{+barrato+}. Il testo
nelle stringhe codice e verbatim non viene processato dalla specifica
sintassi di Org; Viene esportato in verbatim.

\subsection{\LaTeX{} integrato}
\label{sec:orge5e5c4d}
Per le note scentifiche che hanno bisogno di contenere simboli
matematici e per le formule occasionali, Org mode supporta
l'integrazione di codice \LaTeX{} nei sui file. Puoi usare direttamente
una sintassi simile al \TeX{} per simboli speciali, inserire formule e
interi ambienti \LaTeX{}.

\begin{verbatim}
The radius of the sun is R_sun = 6.96 x 10^8 m.  On the other hand,
the radius of Alpha Centauri is R_{Alpha Centauri} = 1.28 x R_{sun}.

\begin{equation}                        % arbitrary environments,
x=\sqrt{b}                              % even tables, figures
\end{equation}                          % etc

If $a^2=b$ and \( b=2 \), then the solution must be
either $$ a=+\sqrt{2} $$ or \[ a=-\sqrt{2} \].
#+end_
\end{verbatim}

\subsection{Esempi letterali}
\label{sec:org513e2ea}
Puoi inserire esempi letterali che non saranno soggetti al
markup. Come gli esempi in monospace, questi sono trattati come i
codici sorgente ed esempi simili.

\begin{verbatim}
#+BEGIN_EXAMPLE
  Some example from a text file.
#+END_EXAMPLE
\end{verbatim}

Per semplicità quando usi un piccolo esempio, puoi anche fare iniziare
la linea dell'esempio con un duepunti seguito da uno spazio. Ci
possono essere anche diversi spazzi dopo il primo spazio vuoto che
segue i duepunti:

\begin{verbatim}
Qui c'é un esempio
  : Qualche esempio per un file di testo.
\end{verbatim}

Se l'esempio é del codice sorgente proveniente da un linguaggio di
programmazione, o ogni altro tipo di testo che possa essere elaborato
da Font Lock in Emacs, puoi chiedere che l'esempio sia mostrato tipo
un buffer Emacs fontificato.

\begin{verbatim}
#+BEGIN_SRC emacs-lisp
  (defun org-xor (a b)
    "Exclusive or."
    (if a (not b) b))
 #+END_SRC
\end{verbatim}

Per modificare l'esempio in un buffer speciale che supporti questo
linguaggio, usa C-' sia per entrare che per uscire dal
buffer di modifica.

\subsection{Immagini}
\label{sec:orgdde4c03}
Un'immagine é un link ad un file immagine che non ha la parte
descrittiva, per esempio

\begin{verbatim}
./img/cat.jpg
\end{verbatim}


Se desideri definire una descrizione per l'imagine e una label per un
riferimento interno ( vedi \ref{sec:org651b8f7} ), assicurati
che il link abbia una linea che lo precede con le parole chiave
\texttt{CAPTION} e \texttt{NAME} come sotto:

\begin{verbatim}
#+CAPTION: This is the caption for the next figure link (or table)
#+NAME:   fig:SED-HR4049
[[./img/a.jpg]]
\end{verbatim}

\subsection{Creare note a piè di pagina}
\label{sec:org41b63bd}
Una nota a piè di pagina é definita in un paragrafo che inizia con un
markatore di footnote tra parentesi quadrate nella colonna 0,
non sono permesse indentazioni. I riferimenti alle note a piè di
pagina sono dei semplici marcatori nelle parentesi quadrate, con
dentro del testo. Per esempio:

\begin{verbatim}
The Org website[fn:1] now looks a lot better than it used to.
...
[fn:1] The link is: https://orgmode.org
\end{verbatim}

I seguenti comandi gestiscono i footnote

\begin{description}
\item[{C-c C-x f (\texttt{org-footnote-action})}] Il comando azione dei footnote. Quando il punto é su un riferimento
al footnote, salta alla definizione. Quando é in una definizione,
salta al (primo) riferimento. Altrimenti, crea un nuovo
footnote. Quando questo comando é chiamato con un argomento
prefisso, un menu aggiuntivo di opzioni, incluso la renumerazione,
viene mostrato.

\item[{C-c C-c (\texttt{org-ctrl-c-ctrl-c})}] Salta tra la definizione e il riferimento.
\end{description}

\section{Esportazione}
\label{sec:orgc8db748}
Org può convertire ed esportare i documenti in un varietà di altri
formati che potresti ritenere meglio strutturati ( vedi \ref{sec:orga976bef} ) e markup ( vedi \ref{sec:org3563413} ) se
possibile.

\subsection{Il dispatcher di esportazione}
\label{sec:org10d3efc}
L'interfaccia di esportazione é l'interfaccia principale per
l'esportazione Org. Un menu gerarchico presenta la configurazione
corrente per i formati di esportazione. Le opzioni vengono mostrate
come semplici bottoni nella stessa schermata.

\begin{description}
\item[{C-c C-e (\texttt{org-export-dispacher})}] Richiama l'interfaccia di esportazione,
\end{description}

Org di default esporta l'intero buffer. Se il buffer Org ha una
regione attiva, allora Org esporta solo quella regione.

\subsection{Impostazioni di esportazione}
\label{sec:org1416194}
L'esportatore riconosce delle linee speciali nel buffer che forniscono
informazioni aggiuntive. Queste linee possono essere inserite in ogni
punto del file:

\begin{verbatim}
#TITLE: I'm in the Mood for Org
\end{verbatim}


La principali opzioni di esportazione includono:

\begin{center}
\begin{tabular}{ll}
\texttt{TITLE} & Il titolo che deve essere mostrato\\[0pt]
\texttt{AUTHOR} & L'autore ( di default viene da \texttt{user-full-name})\\[0pt]
\texttt{DATE} & una data, fissa, oppure un timestamp Org\\[0pt]
\texttt{EMAIL} & email ( di default viene da \texttt{user-mail-address})\\[0pt]
\texttt{LANGUAGE} & il codice di linguaggio, es. \texttt{it}\\[0pt]
\end{tabular}
\end{center}

Il set di opzioni keyword possono essere impostate dal dispatcher (
vedi \ref{sec:org10d3efc} ) usando il comando \texttt{Insert
template} premendo \#

\subsection{Tabelle dei contenuti}
\label{sec:org4d93ca5}
La tabella dei contenuti include tutte le intestazioni del
documento. Va in profondità nello stesso modo di come sono i livelli
delle intestazioni nel file. Se vuoi usare una differente profondità,
o sisabilitarlo per intero, imposta la variabile
\texttt{org-export-with-toc}. Puoi fare la stessa cosa per ogni file, usando
i seguenti elementi \texttt{toc} nella parola chiave \texttt{OPTIONS}:

\begin{verbatim}
#+OPTIONS: toc:2          (include solo due livelli nella TOC)
#+OPTIONS: toc:nil        (non esporta la TOC)
\end{verbatim}

Org normalmente inserisce la tabella dei contenuti direttamente prima
del primo titolo nel file.

\subsection{Includere file}
\label{sec:orgec9b04c}
Durante l'esportazione, puoi includere il contenuto di un'altro
file. Per esempio, per includere il tuo file \texttt{.emacs}, puoi usare:

\begin{verbatim}
#+INCLUDE: "~/.emacs" src emacs-lisp
\end{verbatim}


Il primo parametro é il nome del file da includere. Il secondo
parametro, opzionale, specifica il tipo di blocco \texttt{example}, \texttt{export}
o \texttt{src}. Il terzo parametro, anchesso opzionale, specifica il
linguggio del codice sorgente da usare per formattare il
contenuto. Questo é rilevante sia per il tipo di blocchi \texttt{export} che
\texttt{src}.

Puoi vedere il file incluso con C-c '.

\subsection{Linee di commento}
\label{sec:org1e53647}
Le linee che iniziano con zero o più spazzi bianchi seguiti da un \texttt{\#}
e da uno spazzio bianco sono trattati come commenti e, come tali, non
vengono esportati.

Come del resto, le zone circondate da \texttt{\#+BEGIN\_COMMENT}
\ldots{} \texttt{\#+END\_COMMENT} non sono esportate.

In fine, una keyword \texttt{COMMENT} all'inizio di un elemento, ma prima di
un'altra parola chiave o cookie, commenta l'intero sottoalbero. Il
comando sotto aiuta a cambiare lo stato di commento di un titolo.

\begin{description}
\item[{C-c ; (\texttt{org-toggle-comment})}] Attiva/Disattiva la parola chiave \texttt{COMMENT} all'inizio di un
elemento
\end{description}

\subsection{Exportazione ASCII/UTF-8}
\label{sec:orgc4a76ca}
L'esportazione ASCII produce in output un file di soli caratteri
ASCII. Questo é il più semplice e diretto testo di output. Non
contiene nessun markup Org. L'esportazione UTF-8 usa caratteri
aggiuntivi e simboli avviabili in questo standard di codifica.

\begin{description}
\item[{C-c C-e t a, C-c C-e t u (\texttt{org-ascii-export-to-ascii})}] Esporta come file ASCII con estensione \texttt{.txt}. Con \texttt{myfile.org} Org
esporta \texttt{myfile.txt}, riscrivendolo senza warning. Per \texttt{myfile.txt},
Org esporta in \texttt{myfile.txt.txt} per prevenire il rischio di perdita
dei dati.
\end{description}

\subsection{Esportazione HTML}
\label{sec:org9ee3273}
La modalità Org contiene un esportatore in HTML con estensione HTML e
formatta compatibile con XHTML 1.0 strict standard.

\begin{description}
\item[{C-c C-e h h (\texttt{org-html-export-to-html})}] Esporta con file HTML con estensione \texttt{.html}. Per \texttt{myfile.org}, Org
lo esporta come \texttt{myfile.html}, sovrascrivendolo senza
warning. C-c C-e h o esporta in HTML e apre il file nel
browser web.
\end{description}

Il backend di esportazione trasforma le \texttt{<} e \texttt{>} in \texttt{\&lt;} e \texttt{\&gt;}.
Per inserire del puro codice HTML nel file Org in modo che il backend
di esportazione possa includerlo nell'output, usa questa sintassi in
linea: \texttt{@@html:...@@}. Per esempio:

\begin{verbatim}
@@html:<b>@@bold text@@html:</b>@@
\end{verbatim}


Per un blocco di codice HTML puro più ampio usa questo blocco di
codice di esportazione:

\begin{verbatim}
#+HTML: Literal HTML code for export

#+BEGIN_EXPORT html
  Tutte le linee tra questi marker sono esportati letteralmente.
#+END_EXPORT
\end{verbatim}

\subsection{Esoprtazione \LaTeX{}}
\label{sec:org0dd162a}
L'esportazione \LaTeX{} può elaborare documenti complessi, incorporando
classi \LaTeX{} standar o personalizzate, genera documenti usando engine
\LaTeX{} alternativi, e producendo un un file PDF con indici,
bibliografie e tabelle dei contenuti, destinato alla visione
interattiva o per la pubblicazione di stampa di alta qualità.

Di default, l'output \LaTeX{} usa la classe \emph{article}. Puoi cambiarlo
aggiungendo, nel tuo file, un opzione tipo questa \texttt{\#+LATEX\_CLASS:
myclass}. Le classi possono essere inserite in \texttt{org-latex-classes}.

\begin{description}
\item[{C-c c-e l l (\texttt{org-latex-export-to-latex})}] Esporta in un file \LaTeX{} con estensione \texttt{.tex}. Per \texttt{myfile.org},
Org lo esporta in \texttt{myfile.tex}, sovrascrivendo senza warning.

\item[{C-c C-e l p (\texttt{org-latex-esport-to-pdf})}] Esposta come file \LaTeX{} e lo converte in file PDF.

\item[{C-c C-e l o (\texttt{<no corresponding named command>})}] Esporta come file \LaTeX{} e lo converte in PDF, dopo di che apre il
file PDF usando il visualizzatore di default.
\end{description}

Il backend di esportazione \LaTeX{} può inserire codice \LaTeX{} arbitrario,
vedi \ref{sec:orge5e5c4d}. Ci sono tre modi per inserire del codice nel
file Org e ogniuno usa una sintassi di quotazione diversa.

Inserimento in linea quota con i simboli @:

\begin{verbatim}
Code embedded in-line @@latex:any arbitrary LaTeX code@@ in a paragraph.
\end{verbatim}


Inserendo una o più linee con keyword nel file Org:

\begin{verbatim}
#+LATEX: any arbitrary LaTeX code
\end{verbatim}


Inserendo un blocco di esportazione nel file Org, dove il beckend
esporterà tutto il codice tra il marcatore di inizio e fine:

\begin{verbatim}
#+BEGIN_EXPORT latex
  any arbitrary LaTeX code
#+END_EXPORT
\end{verbatim}

\subsection{Esportazione iCalendar}
\label{sec:org38863d6}
L'enorme successo dell'interoperabilità della modalità Org é l'abilità
di esportare ed importare da applicazioni esterne. Il backend di
esportazione iCalendar prende le date di calendario dai file Org e le
esporta nel formato standard iCalendar.

\begin{description}
\item[{C-c C-e c f (\texttt{org-icalendar-export-to-ics})}] Crea un elemento iCalendar dal buffer Org corrente e lo memorizza
nella stessa directory, usando un file con estensione \texttt{.ics}.

\item[{C-c C-e c c (\texttt{org-icalendar-combine-agenda-files}) }] Crea un file iCalendar combinando i file Org provenienti da
\texttt{org-agenda-files} e scrivendoli nel nome di file
\texttt{org-icalendar-combined-agenda-file}.
\end{description}

\section{Pubblicare}
\label{sec:org3d565e7}
Org include un sistema di gestione delle pubblicazione che ti permette
di configurare una conversione HTML automatica dei \emph{proggeti} composta
dai file Org collegati internamente. Puoi anche configurare Org in
modo che carichi in outomatico le pagine HTML e relativi allegati,
come immagini e file di codice sorgente, in un server web.

Puoi anche usare Org per convertire i file in PDF, o anche combinando
la conversione HTML e PDF in modo che questi file siano disponibili in
entrambi i formati sul server.

Per delle istruzioni dettagliate sulle impostazioni, vedi
\url{manual}. Qui c'é un esempio:

\begin{verbatim}
(setq org-publish-project-alist
      '(("org"
         :base-directory "~/org/"
         :publishing-function org-html-publish-to-html
         :publishing-directory "~/public_html"
         :section-numbers nil
         :with-toc nil
         :html-head "<link rel=\"stylesheet\"
                    href=\"../other/mystyle.css\"
                    type=\"text/css\"/>")))
\end{verbatim}

\begin{description}
\item[{C-c C-e P x (\texttt{org-publish})}] Mostra un prompt per uno specifico progetto e pubblica tutti i file
che lo riguardano.

\item[{C-c C-e P p (\texttt{org-publish-current-project})}] Pubblica il progetto che contiene il file corrente.

\item[{C-c C-e P f (\texttt{org-publish-current-file})}] Pubblica solo il file corrente.

\item[{C-c C-e P a (\texttt{org-publish-all})}] Pubblica tutti i progetti.
\end{description}

Org usa il timestamp per tenere traccia di quando un file viene
cambiato. Le funziona sopra normalmente pubblicano solo i file
modificati. Puoi bypassare questo e forzare la pubblicazione di tutti i
file dando un argomento prefisso ad ogni comando visto sopra.

\section{Lavorare con il codice sorgente}
\label{sec:org1624326}
Org mode fornisce un numero di funzionalità per lavorare con i codici
sorgenti, incluso la modifica di blocchi di codice nella loro
major-mode, valutazione del blocco di codice, tangling di blocchi di
codice e esportazione di blocchi di codice e del loro risultato in
diversi formati.

Un blocco di codice si forma con questa struttura:

\begin{verbatim}
#+NAME: <name>
#+BEGIN_SRC <language> <switches> <header arguments>
  <body>
#+END_SRC
\end{verbatim}

dove:

\begin{itemize}
\item \texttt{<name>} é una stringa usata per dare un nome univoco al blocco di
codice,
\item \texttt{<language>} specifica il linguaggio del blocco di codice,
es. \texttt{emacs\_lisp}, \texttt{shell}, \texttt{R}, \texttt{python} ecc..
\item \texttt{<switches>} può essere usato per controllare l'esportazione del
blocco di codice
\item \texttt{<header arguments>} può essere usato per controllare diversi
aspetti del comportamento del blocco di codice come dimostrato sotto
\item \texttt{<body>} contiene il codice sorgente.
\end{itemize}

Usa C-c ' per modificare il blocco di codice corrente. Apre
un nuovo buffer di modifica nella major-mod contenente il sorgente del
blocco di codice, pronto per ogni modifica. Usando C-c '
nuovamente chiude il buffer e torna al buffer Org.

\subsection*{Usare argomenti di intestazione}
\label{sec:orgdf5a339}
Un argomento di intestazione viene specificato con un carattere
duepunti iniziale seguito nome dall'argomento in minuscolo.

Gli argomenti di intestazione possono essere impostati in diversi
modi; Org gli assegna delle priorità per evitare che ci siano delle
sovrapposizioni o dei conflitti dando alle impostazioni locali la
priorità più alta.

\begin{description}
\item[{Argomenti di intestazione a livello di sistema}] Questi sono specificati pesonalizzando la variabile
\texttt{org-bable-default-header-args}, o, per uno specifico linguaggio
LANG \texttt{org-babel-default-header-args:LANG}.

\item[{Argomenti di intestazione nelle proprietà}] Puoi impostarlo usando la proprietà \texttt{header-args} ( vedi \ref{sec:org306bfd2}
)---o \texttt{header-args:LANG} per linguaggio LANG. Gli
argomenti di intestazione nelle proprietà vengono applicati nei
sotoalberi dei livelli sotto.

\item[{Argomenti di intestazione nei blocchi di codice}] Gli argomenti di intestazione sono comunemente usati per impostare
il sorgente dei blocchi di codice, nella linea \texttt{BEGIN\_SRC}:

\begin{verbatim}
#+NAME: factorial
#+BEGIN_SRC haskell :results silent :exports code :var n=0
  fac 0 = 1
  fac n = n * fac (n-1)
#+END_SRC
\end{verbatim}

Gli argomenti di intestazione nei blocchi di codice possono essere
su più linee usando la parola chiave \texttt{HEADER} su ogni linea.
\end{description}

\subsection*{Valutare i blocci di codice}
\label{sec:orgee6c181}
Usa C-c C-c per valutare il blocco di codice corrente ed
inserire il risultato nel documento Org. Di default, é abilitata solo
la valutazione dei blocchi di codice \texttt{emacs-lisp}, comunque esiste il
supporto per la valutazione di blocchi di codice in diversi
linguaggi. Per una lista completa dei linguaggi suportati guarda il
\href{manual}{manuale}. Di seguito viene mostrato un blocco di codice ed il suo
risultato.

\begin{verbatim}
#+BEGIN_SRC emacs-lisp
  (+ 1 2 3 4)
#+END_SRC

#+RESULTS:
: 10
\end{verbatim}

La sintasso qui sotto viene usata per passare un argomento al blocco
di codice usando l'argomento di intestazione \texttt{var}.

\begin{verbatim}
:var NAME=ASSIGN
\end{verbatim}


NAME é il nome della variabile usata nel corpo del
blocco di codice. ASSIGN é il valore letterale, come una
stringa, un numero, un riferimento ad una tabella, una lista, un
esempio letterale, un'altro blocco di codice---con o senza argomenti
---o il risultato della valutazione di un blocco di codice.

\subsection*{Risultato della valutazione}
\label{sec:org5d7f2fa}
Il modo in cui Org gestisce i risultati dell'esecuzione di un blocco
di codice dipende da come i molti argomenti dell'intestazione lavorano
assieme. Il fattore determinante, tuttavia, è l'argomento
dell'intestazione \texttt{result}. Lui controlla la \emph{collezione}, \emph{tipo},
\emph{formato} e \emph{manipolazione} dei risultati dei blocchi di codice.

\begin{description}
\item[{Collezione}] Come il risultato debba essere collezionato dal blocco di
codice. Puoi scegliere tra \texttt{output} o \texttt{value} (il default).

\item[{Tipo}] Che tipo di risultato mi aspetto dall'esecuzione del blocco di
codice. Puoi scegliere tra \texttt{table}, \texttt{list}, \texttt{scalar} e \texttt{file}. Org
prova ad indovinare se non lo hai impostato.

\item[{Formaato}] Come Org processa il risultato. Alcuni possibili valori sono \texttt{code},
\texttt{drawer}, \texttt{html}, \texttt{latex}, \texttt{link} e \texttt{raw}.

\item[{Maipolazione}] Come viene inserito il risultato una volata formattato
appropriatamente. I valori consentiti sono \texttt{silent}, \texttt{replace} ( il
default ), \texttt{append} o \texttt{prepend}.
\end{description}

I blocchi di codice con i risultati di output in file---es.:
grafici, diagrammi e figure---possono accettare un argomento di
intestazione \texttt{:file FILENAME}, in questo caso i risultati sono salvati
nel file nomitato, e un link al file viene inserito nel buffer.

\subsection*{Esportazione dei blocchi di codice}
\label{sec:org172dff4}
É possibile esportare il \emph{codice} del blocco di codice, il \emph{risultato}
della valutazione del blocco di codice, \emph{entrambe} il codice ed il
risultato della valutazione del blocco di codice, o \emph{niente}. Org di
default, per diversi linguaggi, esporta il \emph{codice}.

L'argomento di intestazione \texttt{exports} server ad informare che quella
parte di codice viene esportata, per esempio, nel formato HTML o
\LaTeX{}. Può essere impostato a \texttt{code}, \texttt{results}, \texttt{both} o \texttt{none}.

\subsection*{Estrazione del codice sorgente}
\label{sec:orgffc8e33}
Usa C-c C-v t per creare un file con solo il codice
sorgente estratto dal blocco di codice del buffer corrente. Questo
viene definito come ``tangling''---un termine adottato dalla comunità di
programmazione letterale. Durante il tangling del blocco di codice il
suo corpo viene espanso usando \texttt{org-balble-expand-src-block}, che può
espandere sia i riferimenti di tipo variabile che quelli di tipo
``Noweb''. Per fare il tangle di un blocco di codice, il blocco, deve
avere un argomento di intestazione \texttt{tangle}, vedi il \href{manual}{manuale} per
maggiori dettagli.

\section{Miscelanee}
\label{sec:org62c1089}
\subsection*{Completion}
\label{sec:org2ef7f0e}
Org ha il completamento nel buffer con M-TAB. Il minibuffer
non é coinvolto. Digita una o più lettere per poi usare la
combinazione di scelta rapida per completare il testo sul posto.

Per esempio, questo comando completerà i simboli Tex dopo \texttt{\textbackslash{}}, le
keyword TODO all'inizio di una intestazione e i tag dopo i \texttt{:} nelle
intestazioni.

\subsection*{Modelli di struttura}
\label{sec:orgcea2100}
Per inserire velocemente dei blocchi strutturati vuoti, come
\texttt{+BEGIN\_SRC} \ldots{} \texttt{\#+END\_SRC}, o per racchiudere il testo in un blocco,
usa:

\begin{description}
\item[{C-c C-, (\texttt{org-insert-structure-template})}] Mostra un prompt per i tipi di strutture a blocchi, e inserisce il
blocco nel punto. Se la regione é attiva, la racchiude nel blocco.
\end{description}

\subsection*{Viste pulite}
\label{sec:orge7a72b1}
L'outline di default di Org con asterischi e senza indentazione può
andare bene per piccoli documenti. Per lunghi documenti come \emph{libri},
l'effetto non é evidente. Org fornisce un'alternativa per gli
asterischi e uno schema di indentazione, come mostrato alla destra
della seguente tabella. Usa un solo asterisco e indenta il testo per
allinearlo al'intestazione:

\begin{verbatim}
* Top level headline             |    * Top level headline
** Second level                  |      * Second level
*** Third level                  |        * Third level
    some text                     |          some text
*** Third level                  |        * Third level
    more text                     |          more text
* Another top level headline     |    * Another top level headline
\end{verbatim}

Questo tipo di visualizzazione può essere ottenuta dinamicamente sul
display usando la modalità Org Indent ( M-x org-indent-mode RET ), che inserisce uno spazio intangibile su ogni riga. Puoi
attivare la modalità Org Indent per tutti i file personalizzando la
variabile \texttt{org-startup-indented}, oppure puoi attivarlo in ogni file
usando:

\begin{verbatim}
#+STARTUP: indent
\end{verbatim}


Se si desidera che l'indentazione sia costituita da caratteri di
spazio fisso, in modo che il file di testo semplice appaia il più
possibile simile alla visualizzazione di Emacs. Org aiuta a rientrare
( con TAB ) il testo sotto ogni titolo, a nascondere le
stelle iniziali e a usare solo i livelli 1, 3, ecc.., per ottenere un
rientro di due caratteri per ogni livello. Per ottenere questo in un
file usa

\begin{verbatim}
#+STARTUP: hidestarts odd
\end{verbatim}

\section{Traduzione}
\label{sec:orgd57a2c3}
Il titolo originale di questo documento é \textbf{Org Mode Compact Guide} che potete
trovare all'indirizzo \url{https://orgmode.org/orgguide.html}, la mia é una
traduzione non ufficiale di tale documento.
\end{document}
